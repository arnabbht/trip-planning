\section{Related Work}
\label{Rel_Work}

We have summarized the recently published relevant papers in Table~\ref{tab:otherworks}. We can see that \trip is the only multi-day itinerary planning solution to offer multimodal transportation, choice of multiple utility variants, and dynamic re-planning. The ability to handle personalized constraints in the form of must-see POIs, must-avoid POIs, category constraints, and ordering constraints makes it even better. We next discuss these works.
 
Chen et al.~\cite{chen2014automatic} gave a scalable method to plan multiday trips according to user preferences. The system returns itineraries with low computation time, trading off scalability and personalization for real-time planning. van Zelst \cite{vanzelst2016itinerary} proposed Integer Linear Programming (ILP) based individualized travel itineraries in urban environments. Taylor et al. \cite{taylor2018tour} studied the problem of integrating user-defined POIs in travel planning. They introduced the \emph{TourMustSee} problem, which is a variation of the well-known Orienteering Problem \cite{golden1987orienteering}. They have considered maximizing utility function similar to our \emph{\trip} solution. However, they have considered only the walking mode of transport and did not factor in the cost budget. Vu et al.~\cite{vu2022branch} advanced the formulation of the Tourist Trip Design Problem (TTDP) by including multiple practical constraints but could not achieve multiple-period itineraries. Multiday itinerary planning solution have been studied by~\cite{chen2014automatic,vanzelst2016itinerary,liu2024personalized,rambha2024optimized}. However, rather than optimizing any utility functions, they optimize either cost or time.  

Rambha ~\cite{panagiotakis2024expectation} presented a deterministic solution using Expectation-Maximization (EM) to construct personalized trips. While their system is a good starting point for static Personalized Itinerary Recommendation (PIR), it lacks some practical considerations applicable to real-world systems. Specifically, the system does not dynamically update the itinerary based on real-time information, and does not support multiple transport modes. A holistic satisfaction model for tour itinerary recommendation was proposed in~\cite{liu2024personalized}. Our extension builds upon this basis with the addition of dynamic real-time travel data collected through Google APIs, facilitating adaptive itinerary revision; fractional POI visit support for light interaction; and multimodal travel modes for more realistic tourist mobility modeling. 

A solution for optimizing the costs of such itineraries using an integer programming model is proposed by~\cite{rambha2024optimized}. Although this work focuses on minimizing the cost budget, it lacks support for handling personalized constraints and dynamic itinerary generation. Another solution PERSTOUR personalizes the trip recommendation for tourists based on user interests, duration and frequency of visit~\cite{lim2018personalized}. Though they could consider POI popularity and user interest preferences, uncertainties in POI visit duration and arrival times have not been addressed. \cite{bolzoni2014efficient} proposed a more realistic approach by adding categories to POIs with maximal cardinality constraints, which, however, falls significantly short in handling multiple real-world constraints.

Detailed reviews of previous work are available in~\cite{gavalas2014survey} and~\cite{sylejmani2011survey}.
Other notable related works include~\cite{zheng2021novel,yu2017mining,jiaoman2018travel,sylejmani2017planning,zografos2008algorithms,rani2018development,yu2014optimal,arora2024itinerary}. In contrast to earlier work, our proposed \trip solution is powered by a wide variety of superior planning capabilities such as multimodal transportation, multiday itinerary planning, choice of utility variants, dynamic adjustment of itinerary and ability to accommodate user-specified personalized constraints.

