\section{Related Work}

\begin{table*}[t]
\centering
\begin{adjustbox}{width=\textwidth,center}
\begin{tabular}{lcccccccccccc}
\toprule

& \bf \makecell{Multi-day\\Trips}
& \bf \makecell{Multi-modal\\Trips}
& \bf \makecell{Dynamic\\Itineraries}
& \bf \makecell{No. of Utility\\Variants}
& \bf \makecell{Time\\Budget} 
& \bf \makecell{Cost\\Budget} 
& \bf \makecell{POI\\Timings} 
& \bf \makecell{Must-see\\POIs} 
& \bf \makecell{Must-avoid\\POIs}
& \bf \makecell{Category\\Constraint} 
& \bf \makecell{Ordering\\Constraint}\\
\midrule
%\midrule
\cite{chen2014automatic}      & \cmark & \xmark & \xmark & 0 & \cmark  &  \xmark & \xmark &  \cmark & \xmark & \xmark & \xmark\\
%\midrule
\cite{vanzelst2016itinerary}  & \cmark  & \xmark & \xmark  & 0 & \xmark & \cmark & \cmark  & \xmark  &  \xmark & \cmark & \xmark \\
%\midrule
\cite{taylor2018tour}         & \xmark & \xmark & \xmark  & 1  & \cmark & \xmark  & \xmark &  \cmark & \xmark  & \xmark & \xmark \\
%\midrule
\cite{vu2022branch}           & \xmark & \xmark &  \xmark & 0  & \cmark & \cmark & \cmark & \cmark & \cmark  & \cmark & \cmark &\\
%\midrule
\cite{panagiotakis2024expectation}      & \xmark & \xmark & \xmark & 0  & \cmark  & \xmark & \xmark & \cmark &  \xmark & \cmark & \cmark &\\
%\midrule
\cite{liu2024personalized}     & \cmark & \xmark & \xmark  & 0 & \cmark & \xmark & \cmark & \cmark   & \xmark &  \cmark  & \xmark\\
%\midrule
\cite{rambha2024optimized}  & \cmark & \xmark & \xmark  &  0  & \xmark  & \cmark  & \cmark &  \xmark & \xmark & \xmark & \xmark &\\
%\midrule
\cite{lim2018personalized}    & \xmark & \xmark & \xmark  & 0  & \cmark  & \cmark & \cmark &  \cmark & \xmark & \cmark & \xmark \\
%\midrule
\cite{bolzoni2014efficient}    & \xmark & \xmark & \xmark  & 2   & \cmark & \xmark & \xmark & \xmark & \xmark & \cmark & \xmark \\
%\midrule
\midrule
\bf {\trip}             & \cmark & \cmark & \cmark & 3  & \cmark & \cmark & \cmark & \cmark & \cmark & \cmark & \cmark & \\
\bottomrule
\end{tabular}
\end{adjustbox}
\caption{Comparison of recent work addressing the trip planning problem}
\label{tab:otherworks}
\end{table*}

This survey is the backdrop for understanding how our approach is different and makes a contribution to the research work carried out in this area. A detailed survey is provided by~\cite{gavalas2014survey} and~\cite{sylejmani2011survey}. ~\cite{chen2014automatic} presents a scalable method to plan multi-day trips from user preferences. The system returns itineraries with much less computation time, trading off scalability and personalization for real-time planning. The use of Integer Linear Programming (ILP) to create individualized travel itineraries in urban environments is illustrated in~\cite{vanzelst2016itinerary}. The problem of integrating user-defined POIs in travel planning is shown by~\cite{taylor2018tour}. They introduce the \textbf{TourMustSee} problem, a variation of the well-known Orienteering Problem. ~\cite{vu2022branch} explored an advanced formulation of the Tourist Trip Design Problem (TTDP) by including multiple real-world constraints.

~\cite{panagiotakis2024expectation} presented a deterministic solution using Expectation-Maximization (EM) to construct personalized trips. While their system is a good starting point for static Personalized Itinerary Recommendation (PIR), it lacks some practical considerations applicable to real-world systems. Specifically, the system does not support fractional visits to POIs, supports static travel time estimates, does not dynamically update the itinerary based on real-time information, and does not support multiple transport modes. A holistic satisfaction model for tour itinerary recommendation was proposed in~\cite{liu2024personalized}. Our extension builds upon this basis with the addition of dynamic real-time travel data collected through Google APIs, facilitating adaptive itinerary revision; fractional POI visitation support for light interaction; and multimodal travel modes for more realistic tourist mobility modeling. 

A solution for optimizing costs of such itineraries using an integer programming model is proposed by~\cite{rambha2024optimized}. Although this work focuses on minimizing the cost budget, it lacks in handling category and ordering constraints, must-see POIs, and dynamic trip planning etc., Another solution PERSTOUR personalizes trip recommendation for tourists based on user interests, points of interest, visit durations and visit recency~\cite{lim2018personalized}. Many features including dynamic itineraries, ordering constraint and multi day/modal trips etc., have not been addressed. Moreover, the data used for this work is restricted to a collection of images which may not be feasible to collect in all environments.~\cite{bolzoni2014efficient} proposed a more realistic approach to trip planning by adding category information to POIs which could capture only time budget, fractional visit and category constraints. 

~\cite{sylejmani2011survey} provide a complete overview of existing trip planning systems, for instance, City Trip Planner, YourTour, Plnnr, and Mtrip as outlined in Table~\ref{tab:otherworks}. Responsiveness and flexibility in future trip planning systems are their findings. CityTripPlanner and YourTour concentrate on generating simple itineraries but fail to consider dynamic scenarios and utility constraints. Mtrip has offline map capabilities and real-time flexibility but lacks planning functionalities such as the operational timings of POIs and category-based constraint.

Few other related work includes~\cite{zheng2021novel},~\cite{yu2017mining},~\cite{jiaoman2018travel},~\cite{sylejmani2017planning},~\cite{zografos2008algorithms},~\cite{rani2018development},~\cite{yu2014optimal} and~\cite{arora2024itinerary}. Our research (2025) offers an \emph{integrated optimization model} covering all the necessary dimensions: \emph{multi-modal transport modes, fractional visiting of POIs, real-time responsiveness, and advanced planning constraints like working days and category limits}. 

\ignore{

\begin{table}[t]
\centering
\begin{adjustbox}{max width=\columnwidth}
\begin{tabular}{lccccc}
\toprule
\textbf{Feature} & \textbf{citytripplanner} & \textbf{yourtour} & \textbf{plnnr} & \textbf{mtrip} & \textbf{\trip} \\
\midrule
Personal Interest Modeling          & \cmark & \cmark & \cmark & \cmark & \cmark \\
Automatic POI Selection             & \cmark & \cmark & \cmark & \cmark & \cmark \\
Followable Itinerary                & \cmark &        & \cmark & \cmark & \cmark \\
Mandatory POIs                      & \cmark & \cmark & \cmark & \cmark & \cmark \\
Max per Category Limit              &        &        &        &        & \cmark \\
Required Category Inclusion         &        &        &        &        & \cmark \\
Routing Through Mandatory POIs      & \cmark & \cmark & \cmark & \cmark & \cmark \\
Navigation Support                  & \cmark & \cmark & \cmark & \cmark & \cmark \\
POI Opening Hours                   & \cmark &        &        &        & \cmark \\
Weather-aware Travel Time           &        &        &        &        & \cmark \\
Context-awareness (Current Location)&        &        &        & \cmark & \cmark \\
POI Popularity Consideration        &        &        & \cmark & \cmark & \cmark \\
\ab{Hotel Selection}                     &        & \cmark & \cmark & \cmark &        \\
Multi-day Trip Planning             & \cmark & \cmark & \cmark & \cmark & \cmark \\
Budget Constraints                  &        & \cmark &        &        & \cmark \\
Adaptive Itinerary Replanning       &        &        &        & \cmark & \cmark \\
\ab{Group Tours and Preferences}         &        & \cmark & \cmark &        &        \\
\bottomrule
\end{tabular}
\end{adjustbox}
\caption{Comparative summary of trip planning systems and our proposed method -- \ab{why two tables?}}
\label{tab:websites}
\end{table}

}

\ignore{

Our system surmounts these limitations with the incorporation of a broad spectrum of state-of-the-art planning capabilities including real-time responsiveness, cost and time budgeting, priority-based POI inclusion, category filtering, and fractional visit support. Multimodal travel guidance and visit ordering constraints are also included. Group planning and lodging logistics cannot yet be done by the system but does excellent work generating highly personalized, constraint-based itineraries for solo travelers seeking optimized and flexible experiences.

}
