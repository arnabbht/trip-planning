\section{Conclusions and Future Work}
\label{Conclusions}

This paper revisits the trip itinerary planning problem and proposes a novel
solution framework called \trip that returns an \emph{optimal itinerary} as a
solution for practical scenarios. The \trip framework allows multi-day itinerary
planning that not only considers the tourist's starting and ending locations and
timings, but also adheres to the operational timings of each POI. The proposed
solution is unique and powerful due to its ability to accommodate multiple
transportation modes, factoring of user-specified personalized constraints such
as must-see constraints, must-avoid constraints, ordering constraints and
category constraints, consideration of multiple utility variants, and the
capability of dynamic re-planning of the itinerary to account for unplanned
delays or early exits experienced during the previously visited POIs. Empirical
evaluation on several popular destinations show the efficacy and efficiency of
the proposed solution.

In future, inclusion of other transportation modes such as cycle, public bus and
metro can be considered. Further, real-time crowd and resulting waiting times at
each POI can be factored to generate more useful itineraries.

