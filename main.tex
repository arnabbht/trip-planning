%% For submission and review of your manuscript please change the
%% command to \documentclass[manuscript, screen, review]{acmart}.
%%
%% When submitting camera ready or to TAPS, please change the command
%% to \documentclass[sigconf]{acmart} or whichever template is required
%% for your publication.
%%
%%
%\documentclass[sigconf,anonymous,review]{acmart}
\documentclass[conference]{IEEEtran}

%% \BibTeX command to typeset BibTeX logo in the docs
%\AtBeginDocument{%
%  \providecommand\BibTeX{{%
%    Bib\TeX}}}
\def\BibTeX{{\rm B\kern-.05em{\sc i\kern-.025em b}\kern-.08em
    T\kern-.1667em\lower.7ex\hbox{E}\kern-.125emX}}

%%% Rights management information.  This information is sent to you
%%% when you complete the rights form.  These commands have SAMPLE
%%% values in them; it is your responsibility as an author to replace
%%% the commands and values with those provided to you when you
%%% complete the rights form.
%\setcopyright{acmlicensed}
%\copyrightyear{2018}
%\acmYear{2018}
%\acmDOI{XXXXXXX.XXXXXXX}
%%% These commands are for a PROCEEDINGS abstract or paper.
%\acmConference[Conference acronym 'XX]{Make sure to enter the correct
%  conference title from your rights confirmation email}{June 03--05,
%  2018}{Woodstock, NY}
%%%
%%%  Uncomment \acmBooktitle if the title of the proceedings is different
%%%  from ``Proceedings of ...''!
%%%
%%%\acmBooktitle{Woodstock '18: ACM Symposium on Neural Gaze Detection,
%%%  June 03--05, 2018, Woodstock, NY}
%\acmISBN{978-1-4503-XXXX-X/2018/06}
%
%
%%%
%%% Submission ID.
%%% Use this when submitting an article to a sponsored event. You'll
%%% receive a unique submission ID from the organizers
%%% of the event, and this ID should be used as the parameter to this command.
%%%\acmSubmissionID{123-A56-BU3}
%
%%%
%%% For managing citations, it is recommended to use bibliography
%%% files in BibTeX format.
%%%
%%% You can then either use BibTeX with the ACM-Reference-Format style,
%%% or BibLaTeX with the acmnumeric or acmauthoryear sytles, that include
%%% support for advanced citation of software artefact from the
%%% biblatex-software package, also separately available on CTAN.
%%%
%%% Look at the sample-*-biblatex.tex files for templates showcasing
%%% the biblatex styles.
%%%
%
%%\citestyle{acmnumeric}

\usepackage{cite}
\usepackage{algorithmic}
\usepackage{amsmath}
\usepackage{amssymb}
\usepackage{amsfonts}
\usepackage{hyperref}
\usepackage{url}
%\usepackage{tcolorbox} 
%\usepackage{fullpage}
\usepackage{adjustbox}
%\usepackage{pifont}
\usepackage{makecell}
%\usepackage{float}  % in preamble
\usepackage{graphicx}
\usepackage{comment}
\usepackage{enumitem}
%\usepackage{floatrow}
\usepackage{textcomp}
\usepackage{wrapfig}
%\usepackage{lscape}
%\usepackage{rotating}
\usepackage{graphicx}
\usepackage{caption}
%\usepackage{upgreek}
\usepackage{gensymb}
\usepackage{tabularx}
%\usepackage{csquotes}
%\usepackage{pdfpages}
%\usepackage{lipsum}
\usepackage{booktabs}
%\usepackage{array}
%\usepackage{adjustbox}
%\usepackage{amssymb}
\usepackage{pifont}
\usepackage{xspace}
\usepackage{xcolor}
\usepackage{float}

\usepackage{fontawesome}
\usepackage{scalefnt}

\setlist{nosep,leftmargin=*}

\newcommand{\trip}{TRIP\xspace}
\definecolor{darkgreen}{RGB}{0,150,0}
\newcommand{\cmark}{{\color{darkgreen}\ding{51}}}

\definecolor{darkred}{RGB}{200,0,0}
\newcommand{\xmark}{{\color{darkred}\ding{55}}}

%\newcommand{\comment}[1]{}
\newcommand{\ignore}[1]{}

\newcommand{\ab}[1]{\textcolor{red}{Arnab: #1}}
\newcommand{\ari}[1]{\textcolor{blue}{Arindam: #1}}
\newcommand{\sm}[1]{\textcolor{blue}{Shubhadip: #1}}
\newcommand{\nl}[1]{\textcolor{darkgreen}{Neelu: #1}}
\newcommand{\pri}[1]{\textcolor{orange}{Priyanshu: #1}}

\begin{document}

\title
{\scalefont{0.98}TRIP: Optimal Personalized Multi-day Itinerary with Multi-modal Transportation and Dynamic Re-planning}

%\renewcommand{\shorttitle}{Planning the Perfect Itinerary}

%%
%% The "author" command and its associated commands are used to define
%% the authors and their affiliations.
%% Of note is the shared affiliation of the first two authors, and the
%% "authornote" and "authornotemark" commands
%% used to denote shared contribution to the research.

%\author{Anonymous Authors}

\author{
\IEEEauthorblockN{Priyanshu Jha\IEEEauthorrefmark{1}}
\IEEEauthorblockA{\textit{Computer Science and Engineering} \\
\textit{Indian Institute of Technology Kanpur}\\
Kanpur, India \\
priyanshu.jha.1973@gmail.com
0009-0004-1030-4451
}
\and
\IEEEauthorblockN{Neelu Lalchandani\IEEEauthorrefmark{1}}
\IEEEauthorblockA{\textit{Computer Science and Engineering} \\
\textit{Indian Institute of Technology Kanpur}\\
Kanpur, India \\
neelulalchandani29@gmail.com
%ORCID
}
\and
\IEEEauthorblockN{Shubhadip Mitra}
\IEEEauthorblockA{\textit{Blue Yonder India Pvt. Ltd.} \\
\textit{Bengaluru}\\
Bengaluru, India \\
shubhadip.mitra@blueyonder.com}
\and
\IEEEauthorblockN{Arindam Pal}
\IEEEauthorblockA{\textit{TechSoftX Corporation}\\
Sydney, NSW, Australia \\
arindamp@techsoftx.com.au}
\and
\IEEEauthorblockN{Oswald C}
\IEEEauthorblockA{\textit{Computer Science and Engineering} \\
\textit{National Institute of Technology Tiruchirappalli}\\
Tiruchirappalli, India \\
oswald@nitt.edu }
\and
\IEEEauthorblockN{Arnab Bhattacharya}
\IEEEauthorblockA{\textit{Computer Science and Engineering} \\
\textit{Indian Institute of Technology Kanpur}\\
Kanpur, India \\
arnabb@iitk.ac.in
%ORCID
\and
\IEEEauthorblockA{\IEEEauthorrefmark{1}Both authors contributed equally to this
work}
%\and
%\IEEEauthorblockA{\IEEEauthorrefmark{2}Computer}
}
}

\ignore{

\author{Priyanshu Jha}
\affiliation{%
  \institution{Dept. of Computer Science and Engineering}
  \city{IIT Kanpur}
  \country{India}
  \email{priyanshu.jha.1973@gmail.com}
}

\author{Neelu Lalchandani}
\affiliation{%
  \institution{Dept. of Computer Science and Engineering}
  \city{IIT Kanpur}
  \country{India}
  \email{neelulalchandani29@gmail.com}
}

\author{Shubhadip Mitra}
\affiliation{%
  \institution{Blue Yonder India Pvt. Ltd.}
  \city{Bengaluru}
  \country{India}
  \email{shubhadip.mitra@blueyonder.com}
}

\author{Arindam Pal}
\affiliation{%
  \institution{TechSoftX Corporation}
  \city{Sydney}
  \state{NSW}
  \country{Australia}
  \email{arindamp@techsoftx.com.au}
}

\author{Oswald C}
\affiliation{%
  \institution{Dept. of Computer Science and Engineering}
  \city{NIT Tiruchirappalli}
  \country{India}
  \email{oswald@nitt.edu}
}

\author{Arnab Bhattacharya}
\affiliation{%
  \institution{Dept. of Computer Science and Engineering}
  \city{IIT Kanpur}
  \country{India}
  \email{arnabb@iitk.ac.in}
}

}

%\renewcommand{\shortauthors}{Trovato et al.}

%\author{Neelu Lalchandani, Priyanshu Jha, Shubhadip Mitra, Arindam Pal, Oswald C, Arnab Bhattacharya}

\maketitle

\begin{abstract}
	%
	Given a tourist who intends to visit a set of points of interest (POIs)
	spread across a geographical region, the \emph{Trip Itinerary Planning
	(TIP)} problem aims to identify an optimal itinerary that maximizes the
	tourist's utility under a specified cost and time budget. An itinerary is
	defined as an ordered sequence of POIs that adheres to the time and budget
	constraints. This problem is not only important for tourists, but also for
	tour planners that offer personalized trips as business. Although there are
	few prior works that have attempted to address the above problem, they allow
	limited flexibility in terms of accommodating multiple transportation modes,
	re-planning the itinerary based on tourist's actual visit duration of the
	POIs and the live traffic situation, customizing the itinerary based on the
	personalized constraints and the utility function chosen by the user. This
	work revisits the TIP problem with the goal of overcoming the above
	limitations and considering more realistic factors.
	%
	In particular, the proposed solution allows the tourist to (1) choose from a
	set of utility function variants that best captures his/her travel
	experience; (2) factor in multiple transportation modes based on available
	cost and time budget; (3) dynamically adjust the remaining itinerary based
	on the actual time spent till the current POI; (4) plan a multi-day
	itinerary that considers the opening and the closing times of the POIs (in
	addition to open and closed days) and user's choice of the start and the end
	time of the trip and the originating POI and the ending POI for each day;
	(5) customize the itinerary by allowing the tourist to choose from a rich
	set of personalized constraints that include must-visit constraints,
	must-avoid constraints, category constraints and ordering constraints. The
	problem is modeled using a directed graph, where the nodes represent the
	POIs and the edges represent the available transportation modes between each
	pair of POIs. Subsequently, this problem is solved using a mixed integer
	linear program (MILP) that returns the optimal itinerary. A comprehensive
	empirical evaluation over a real data set comprising of several popular
	tourist destinations demonstrate the efficacy and efficiency of the proposed
	solution.
	%
\end{abstract}



%\ccsdesc[500]{Do Not Use This Code~Generate the Correct Terms for Your Paper}
%\ccsdesc[300]{Do Not Use This Code~Generate the Correct Terms for Your Paper}
%\ccsdesc{Do Not Use This Code~Generate the Correct Terms for Your Paper}
%\ccsdesc[100]{Do Not Use This Code~Generate the Correct Terms for Your Paper}

%%
%% Keywords. The author(s) should pick words that accurately describe
%% the work being presented. Separate the keywords with commas.
%\keywords{Itinerary Planning, Graph Algorithms, Mathematical Programming, Optimization, Points of Interest}
%% A "teaser" image appears between the author and affiliation
%% information and the body of the document, and typically spans the
%% page.

\begin{IEEEkeywords}
Itinerary Planning, Graph Algorithms, Mathematical Programming, Optimization, Points of Interest
\end{IEEEkeywords}

\section{Introduction}

Tourism is one of the most dynamic and rapidly growing sectors of the modern
economy. However, planning trips that offer a rich travel experience remains a
non-trivial problem. A tourist planning to visit a new city faces the problem of
planning a suitable \emph{itinerary} that maximizes her utility under a given
cost budget and time budget. Here, an itinerary refers to a sequence of
\emph{points of interest (POIs)}, along with the respective time of arrival and
departure for each POI such that each POI is visited at most once. With the
increasing number of tourists and the increased availability of spatio-temporal
data, there is growing research interest in planning the trip itinerary
\cite{li2016travel, gavalas2014survey, sylejmani2011survey}. This problem is
important not only for tourists, but also for tour planners that offer
personalized trips.

Traditionally, most tourist destinations offer a set of pre-defined itineraries
that do not necessarily fit directly within the tourists' schedule, cost budget,
and preferences. As digital tourism resources and urban mobility platforms grow,
tourists expect personalized, optimal, and efficient travel itineraries that
meet their needs, constraints, and resources. Such itineraries are challenging
to construct manually, due to the underlying complexity arising, due to the number
of POIs, varying travel costs and durations, entrance fees, opening hours, and
different user interests. This creates a need for intelligent
itinerary planning systems that can generate effective and personalized trip
itineraries that offer high utility, while being cognizant of the requirements
of tourists.

Although there exist several prior works \cite{chen2014automatic,
vanzelst2016itinerary, taylor2018tour, vu2022branch,
panagiotakis2024expectation, liu2024personalized, rambha2024optimized,
lim2018personalized, bolzoni2014efficient}, they have several key limitations. (1)~None of the existing works allow the tourist to
choose from multiple available transportation modes such as walking, taxi, or a
hybrid mode allowing usage of multiple transportation modes
\cite{chen2014automatic, taylor2018tour, vanzelst2016itinerary}. (2)~Many
tourists spend more than a day at a given tourist destination. This scenario
ideally calls for a multi-day trip itinerary planning that is aware of opening
and closing days of each POI along with opening and closing time for each open
day. However, majority of the earlier studies focus only on single-day itinerary
planning \cite{taylor2018tour, vu2022branch, panagiotakis2024expectation}.
While it may be possible to combine multiple single day itineraries to generate
a multi-day itinerary, it is not guaranteed to be as cost-effective. (3)~Since most tourists want a rich travel experience, it would be more prudent to optimize the tourists' utility rather than choosing some other objective. However, majority of the earlier studies focussed on optimizing the cost or time,rather than optimizing the user experience \cite{chen2014automatic, vanzelst2016itinerary, liu2024personalized, rambha2024optimized}. (4)~Often there
are tourists with specific requirements and preferences based on individual
priorities and interests, which must be respected,
along with the tourist's cost and time budget constraints. However,
most existing itinerary planning solutions ignore this need to
\emph{personalize} trip itineraries \cite{rambha2024optimized, yu2017mining,
rani2018development, yu2014optimal}. (5)~While traveling based on a planned
itinerary, it is often required to dynamically adjust the remaining itinerary based on
unforeseen delays or altered conditions. However, none of the existing works address this requirement
\cite{chen2014automatic, taylor2018tour, vanzelst2016itinerary}.

Motivated by the above limitations, this work addresses the following itinerary planning problem: \emph{Given a tourist who intends to visit a set of POIs spread across a geographical region, how to identify an optimal itinerary that maximizes the tourist's utility under a specified time and cost budget and respects her choices and availability schedule that may span multiple days} The itinerary must adhere to the tourist's specified starting location and time, and ending location and time for each day of the trip. Additionally, it must consider the operational hours (day and timings) of each POI. The itinerary should factor in the tourist's preferred mode of transportation such as only walking, only taxi, or a hybrid mode that uses both walking and taxi, as necessary. The cost of the itinerary comprises of two components: (a)~\textbf{Transportation cost}, i.e., the cost incurred in traveling, (b)~\textbf{Visiting cost}, i.e., the cost incurred due to entry fees at each POI. Similarly, the time incurred is due to two components: (a)~\textbf{Travel time} between POIs, and (b)~\textbf{Visiting time} in a POI. The tourist may also specify one or more personalized constraints that include the following: (a)~\textbf{Must-see constraints:} These constraints specify a set of POIs that should necessarily be part of the itinerary, (b)~\textbf{Must-avoid constraints:} These constraints specify the set of POIs that must be excluded from the itinerary, (c)~\textbf{Ordering constraints:} These indicate relative ordering between two or more POIs in the itinerary, (d)~\textbf{Category constraints:} Based on the similarity of the POIs, they may be classified into one or more categories. For example, museums, lakes and temples could be a set of categories. The category constraints allow a tourist to specify a lower and an upper bound on the number of POIs she wants to visit from each category. For instance, she may decide to visit at least one but at most two museums. 

\begin{table*}[t]
\centering
\begin{adjustbox}{width=\textwidth,center}
\begin{tabular}{lcccccccccccc}
\toprule
& \bf \makecell{Multi-modal\\Trans.}
& \bf \makecell{Dynamic\\Re-planning}
& \bf \makecell{Multi-day\\Trips}
& \bf \makecell{No. of Utility\\Variants}
& \bf \makecell{Time\\Budget} 
& \bf \makecell{Cost\\Budget} 
& \bf \makecell{POI\\Timings} 
& \bf \makecell{Must-see\\POIs} 
& \bf \makecell{Must-avoid\\POIs}
& \bf \makecell{Category\\Constraints} 
& \bf \makecell{Ordering\\Constraints}\\
\midrule
%\midrule
\bf {\trip}             & \cmark & \cmark & \cmark & 3  & \cmark & \cmark & \cmark & \cmark & \cmark & \cmark & \cmark & \\
\midrule 
\cite{bolzoni2014efficient}    & \xmark & \xmark & \xmark  & 2   & \cmark & \xmark & \xmark & \xmark & \xmark & \cmark & \xmark \\
%\midrule
\cite{chen2014automatic}      & \xmark & \xmark & \cmark & 0 & \cmark  &  \xmark & \xmark &  \cmark & \xmark & \xmark & \xmark\\
%\midrule
\cite{lim2018personalized}    & \xmark & \xmark & \xmark  & 0  & \cmark  & \cmark & \cmark &  \cmark & \xmark & \cmark & \xmark \\
%\midrule
\cite{liu2024personalized}     & \xmark & \xmark & \cmark  & 0 & \cmark & \xmark & \cmark & \cmark   & \xmark &  \cmark  & \xmark\\
%\midrule
\cite{panagiotakis2024expectation}      & \xmark & \xmark & \xmark & 0  & \cmark  & \xmark & \xmark & \cmark &  \xmark & \cmark & \cmark &\\
%\midrule
\cite{rambha2024optimized}  & \xmark & \xmark & \cmark  &  0  & \xmark  & \cmark  & \cmark &  \xmark & \xmark & \xmark & \xmark &\\
%\midrule
\cite{taylor2018tour}         & \xmark & \xmark & \xmark  & 1  & \cmark & \xmark  & \xmark &  \cmark & \xmark  & \xmark & \xmark \\
%\midrule
\cite{vanzelst2016itinerary}  & \xmark  & \xmark & \cmark  & 0 & \xmark & \cmark & \cmark  & \xmark  &  \xmark & \cmark & \xmark \\
%\midrule
\cite{vu2022branch}           & \xmark & \xmark &  \xmark & 0  & \cmark & \cmark & \cmark & \cmark & \cmark  & \cmark & \cmark &\\
%\midrule
\bottomrule
\end{tabular}
\end{adjustbox}
\tabcaption{A summary of key related works based on the solution capabilities.}
\label{tab:otherworks}
\end{table*}

Based on the feedback of previous tourists, each POI is associated with a utility score. 
% assumed to have a user-rating and recommended visit duration. 
% The utility of a tourist at a given POI depends on the fraction of recommended visit duration she actually spends at the given POI and its average user-rating.
The utility of the itinerary is aggregate of the utilities derived at each visited POI. The proposed trip itinerary planning problem allows the tourist to choose from a set of three utility variants that best captures her travel experience. The first variant offers full utility at a POI only if the tourist spends at least the recommended visit duration at the given POI, and zero otherwise. The second variant offers utility that is proportional to the fraction of time that the tourist spends at the given POI w.r.t. its recommended visit duration, provided the tourist  spends at least a minimum specified time. While the first variant can be viewed as a binary step function, the second variant can be viewed as its continuous linear counterpart. The third variant is a multi-step utility function., i.e., a $t$-step utility function where $t \ge 3$. The goal is to maximize the chosen utility variant.

Additionally, if there are unplanned delays or early exits in the earlier part of the itinerary, it should be possible to dynamically update the remaining itinerary based on the remaining cost budget and time budget. Here it is important to note that the reported itinerary not only returns a sequence of POIs to be visited, but also determines the amount of time the tourist spends at each POI (that in turn affects her utility) along with the suitable transportation mode (if more than one transportation modes are available) which in turn affects the travel time and the travel cost of the itinerary.

To address the above problem, this work proposes a novel solution framework,  called \textbf{\trip (TRip Itinerary Planner)}.  Firstly, the problem is modeled as a directed multi-graph $G$, where, each node corresponds to a POI, and each edge corresponds to a travel between an ordered pair of POIs using a specific transportation mode such as walking or taxi. If there are $k$ ($k \ge 1$) transportation modes available between a given pair of POIs, then there are exactly $k$ directed edges between the corresponding pair of nodes in $G$, where each edge corresponds to a specific transportation mode, along with the associated travel cost and travel time. Subsequently, the solution is modeled using a mixed integer linear program (MILP) that returns an optimal itinerary. 

Notably, the given trip itinerary planning problem is NP-hard as it is a strict generalization of the well-known orienteering problem \cite{vansteenwegen2011orienteering, gunawan2016orienteering, vansteenwegen2019orienteering}. Therefore, any optimal solution requires exponential time complexity. However, the itinerary planning problems of practical interest are sufficiently small in size, which allows solving them within acceptable running times (demonstrated in Sec.~\ref{sec:experiments}). Hence, we chose MILP based solution that returns optimal solution within reasonable computation time. This ensures best travel experience under the given constraints.

The major contributions of this work are as follows:

\begin{enumerate}
\item \textbf{Optimal Multi-day Itinerary:} This work proposes a novel multi-day trip itinerary planning solution, named \trip~ that returns an optimal itinerary under the specified cost budget and time budget constraints, while factoring the operational timings of each POI.  
\item \textbf{Multimodal Transportation:} To the best of our knowledge, this is the first work to consider multiple transportation modes such as only walking, only taxi, or a hybrid mode that uses both walking as well as taxi, while planning trip itineraries.
\item \textbf{Personalized Constraints:} The proposed solution framework allows the tourist to specify one or more personalized constraints in the form of must-see constraints, must-avoid constraints, ordering constraints, category constraints, etc.
\item \textbf{Utility Variants:} The \trip~solution framework allows the tourist to choose a utility variant from a set of three utility variants that best models her travel experience, as discussed above. The itinerary that maximizes the chosen utility variant is reported as the solution. This is the first work to consider multiple utility variants while planning \emph{multi-day itinerary}.
\item \textbf{Dynamic Re-planning:} To the best of our knowledge, this is the first work that allows dynamic adjustment of the remaining itinerary based on unplanned delays or early exits, while visiting the previous POIs of the itinerary.
%\item Empirical evaluation on several popular tourist destinations confirm the efficacy and efficiency of the proposed solution. 
%
\end{enumerate}

The rest of the paper is organized as follows. Sec.~\ref{Rel_Work}  discusses the related work. The trip itinerary planning problem is stated in Sec.~\ref{sec:problem}. Sec.~\ref{sec:ilp} presents the \trip solution framework and the experimental results are demonstrated in Sec.~\ref{sec:experiments}. Finally, Sec.~\ref{Conclusions} concludes the paper.
The code and data are available from: \url{https://github.com/priyanshujhaa/TRIP}.


\section{Related Work}

\begin{table*}[t]
\centering
\begin{adjustbox}{width=\textwidth,center}
\begin{tabular}{lcccccccccccc}
\toprule

& \bf \makecell{Multi-day\\Trips}
& \bf \makecell{Multi-modal\\Trips}
& \bf \makecell{Dynamic\\Itineraries}
& \bf \makecell{No. of Utility\\Variants}
& \bf \makecell{Time\\Budget} 
& \bf \makecell{Cost\\Budget} 
& \bf \makecell{POI\\Timings} 
& \bf \makecell{Must-see\\POIs} 
& \bf \makecell{Must-avoid\\POIs}
& \bf \makecell{Category\\Constraint} 
& \bf \makecell{Ordering\\Constraint}\\
\midrule
%\midrule
\cite{chen2014automatic}      & \cmark & \xmark & \xmark & 0 & \cmark  &  \xmark & \xmark &  \cmark & \xmark & \xmark & \xmark\\
%\midrule
\cite{vanzelst2016itinerary}  & \cmark  & \xmark & \xmark  & 0 & \xmark & \cmark & \cmark  & \xmark  &  \xmark & \cmark & \xmark \\
%\midrule
\cite{taylor2018tour}         & \xmark & \xmark & \xmark  & 1  & \cmark & \xmark  & \xmark &  \cmark & \xmark  & \xmark & \xmark \\
%\midrule
\cite{vu2022branch}           & \xmark & \xmark &  \xmark & 0  & \cmark & \cmark & \cmark & \cmark & \cmark  & \cmark & \cmark &\\
%\midrule
\cite{panagiotakis2024expectation}      & \xmark & \xmark & \xmark & 0  & \cmark  & \xmark & \xmark & \cmark &  \xmark & \cmark & \cmark &\\
%\midrule
\cite{liu2024personalized}     & \cmark & \xmark & \xmark  & 0 & \cmark & \xmark & \cmark & \cmark   & \xmark &  \cmark  & \xmark\\
%\midrule
\cite{rambha2024optimized}  & \cmark & \xmark & \xmark  &  0  & \xmark  & \cmark  & \cmark &  \xmark & \xmark & \xmark & \xmark &\\
%\midrule
\cite{lim2018personalized}    & \xmark & \xmark & \xmark  & 0  & \cmark  & \cmark & \cmark &  \cmark & \xmark & \cmark & \xmark \\
%\midrule
\cite{bolzoni2014efficient}    & \xmark & \xmark & \xmark  & 2   & \cmark & \xmark & \xmark & \xmark & \xmark & \cmark & \xmark \\
%\midrule
\midrule
\bf {\trip}             & \cmark & \cmark & \cmark & 3  & \cmark & \cmark & \cmark & \cmark & \cmark & \cmark & \cmark & \\
\bottomrule
\end{tabular}
\end{adjustbox}
\caption{Comparison of recent work addressing the trip planning problem}
\label{tab:otherworks}
\end{table*}

This survey is the backdrop for understanding how our approach is different and makes a contribution to the research work carried out in this area. A detailed survey is provided by~\cite{gavalas2014survey} and~\cite{sylejmani2011survey}. ~\cite{chen2014automatic} presents a scalable method to plan multi-day trips from user preferences. The system returns itineraries with much less computation time, trading off scalability and personalization for real-time planning. The use of Integer Linear Programming (ILP) to create individualized travel itineraries in urban environments is illustrated in~\cite{vanzelst2016itinerary}. The problem of integrating user-defined POIs in travel planning is shown by~\cite{taylor2018tour}. They introduce the \textbf{TourMustSee} problem, a variation of the well-known Orienteering Problem. ~\cite{vu2022branch} explored an advanced formulation of the Tourist Trip Design Problem (TTDP) by including multiple real-world constraints.

~\cite{panagiotakis2024expectation} presented a deterministic solution using Expectation-Maximization (EM) to construct personalized trips. While their system is a good starting point for static Personalized Itinerary Recommendation (PIR), it lacks some practical considerations applicable to real-world systems. Specifically, the system does not support fractional visits to POIs, supports static travel time estimates, does not dynamically update the itinerary based on real-time information, and does not support multiple transport modes. A holistic satisfaction model for tour itinerary recommendation was proposed in~\cite{liu2024personalized}. Our extension builds upon this basis with the addition of dynamic real-time travel data collected through Google APIs, facilitating adaptive itinerary revision; fractional POI visitation support for light interaction; and multimodal travel modes for more realistic tourist mobility modeling. 

A solution for optimizing costs of such itineraries using an integer programming model is proposed by~\cite{rambha2024optimized}. Although this work focuses on minimizing the cost budget, it lacks in handling category and ordering constraints, must-see POIs, and dynamic trip planning etc., Another solution PERSTOUR personalizes trip recommendation for tourists based on user interests, points of interest, visit durations and visit recency~\cite{lim2018personalized}. Many features including dynamic itineraries, ordering constraint and multi day/modal trips etc., have not been addressed. Moreover, the data used for this work is restricted to a collection of images which may not be feasible to collect in all environments.~\cite{bolzoni2014efficient} proposed a more realistic approach to trip planning by adding category information to POIs which could capture only time budget, fractional visit and category constraints. 

~\cite{sylejmani2011survey} provide a complete overview of existing trip planning systems, for instance, City Trip Planner, YourTour, Plnnr, and Mtrip as outlined in Table~\ref{tab:otherworks}. Responsiveness and flexibility in future trip planning systems are their findings. CityTripPlanner and YourTour concentrate on generating simple itineraries but fail to consider dynamic scenarios and utility constraints. Mtrip has offline map capabilities and real-time flexibility but lacks planning functionalities such as the operational timings of POIs and category-based constraint.

Few other related work includes~\cite{zheng2021novel},~\cite{yu2017mining},~\cite{jiaoman2018travel},~\cite{sylejmani2017planning},~\cite{zografos2008algorithms},~\cite{rani2018development},~\cite{yu2014optimal} and~\cite{arora2024itinerary}. Our research (2025) offers an \emph{integrated optimization model} covering all the necessary dimensions: \emph{multi-modal transport modes, fractional visiting of POIs, real-time responsiveness, and advanced planning constraints like working days and category limits}. 

\ignore{

\begin{table}[t]
\centering
\begin{adjustbox}{max width=\columnwidth}
\begin{tabular}{lccccc}
\toprule
\textbf{Feature} & \textbf{citytripplanner} & \textbf{yourtour} & \textbf{plnnr} & \textbf{mtrip} & \textbf{\trip} \\
\midrule
Personal Interest Modeling          & \cmark & \cmark & \cmark & \cmark & \cmark \\
Automatic POI Selection             & \cmark & \cmark & \cmark & \cmark & \cmark \\
Followable Itinerary                & \cmark &        & \cmark & \cmark & \cmark \\
Mandatory POIs                      & \cmark & \cmark & \cmark & \cmark & \cmark \\
Max per Category Limit              &        &        &        &        & \cmark \\
Required Category Inclusion         &        &        &        &        & \cmark \\
Routing Through Mandatory POIs      & \cmark & \cmark & \cmark & \cmark & \cmark \\
Navigation Support                  & \cmark & \cmark & \cmark & \cmark & \cmark \\
POI Opening Hours                   & \cmark &        &        &        & \cmark \\
Weather-aware Travel Time           &        &        &        &        & \cmark \\
Context-awareness (Current Location)&        &        &        & \cmark & \cmark \\
POI Popularity Consideration        &        &        & \cmark & \cmark & \cmark \\
\ab{Hotel Selection}                     &        & \cmark & \cmark & \cmark &        \\
Multi-day Trip Planning             & \cmark & \cmark & \cmark & \cmark & \cmark \\
Budget Constraints                  &        & \cmark &        &        & \cmark \\
Adaptive Itinerary Replanning       &        &        &        & \cmark & \cmark \\
\ab{Group Tours and Preferences}         &        & \cmark & \cmark &        &        \\
\bottomrule
\end{tabular}
\end{adjustbox}
\caption{Comparative summary of trip planning systems and our proposed method -- \ab{why two tables?}}
\label{tab:websites}
\end{table}

}

\ignore{

Our system surmounts these limitations with the incorporation of a broad spectrum of state-of-the-art planning capabilities including real-time responsiveness, cost and time budgeting, priority-based POI inclusion, category filtering, and fractional visit support. Multimodal travel guidance and visit ordering constraints are also included. Group planning and lodging logistics cannot yet be done by the system but does excellent work generating highly personalized, constraint-based itineraries for solo travelers seeking optimized and flexible experiences.

}

\section{The Trip Itinerary Planning Problem}
\label{sec:problem}

We now introduce the \emph{Trip Itinerary Planning (TRIP) Problem}.
We assume a city with different points of interest (POIs),
represented as a graph where each POI is a node.  Every pair of POIs
is connected by multiple edges that represent the mode of transport
from one POI to the other.  Thus, if there are two modes of
transport---walking and taxi---there will be two edges between every
pair of POIs.  The weights on an edge $e$ is a tuple $\langle t_e,
c_e \rangle$ that represent the time of travel and the cost of travel
respectively for the particular travel mode.  The time of travel is
typically measured using a constant speed.  Thus, the distance
between the pair of POIs is divided by the constant speed to convert it
to time.  Similarly, the taxi cost is computed using a fixed formula
on the distance.  For walking, the cost is $0$.  Each POI $v$ has a
tuple $\langle t_v, c_v, U_v \rangle$ associated with it that
represents the time it takes to visit the POI, the cost (entry fees,
etc.) and the utility a traveler gets by visiting the POI
respectively.  While the
time to visit may change from one traveler to another, in this work,
we keep this fixed as the average time.  We do, however, let the time of visit of a POI
vary for a user, with the associated
utility also changing.  Additionally, a POI may have a feasible
time-interval of visit in a day (say, a sunset spot, which can be
visited only in the afternoon from 4-6 pm), beyond which visiting it
gives a utility of $0$.  Each POI is also marked with a category or type (e.g.,
park, museum, etc.).  Finally, we have a total time budget and a cost
budget.  Both the time and the cost budgets are spent in traveling as
well as visiting the POIs.  The objective of the \emph{TRIP} problem
is to \emph{find an itinerary}, i.e., a sequence of POIs, that
\emph{maximizes the total utlity} of visiting the POIs under the
given time and cost budget.  The itinerary is prepended by a source
point (e.g., a hotel or an airport) and appended by a destination
point (again, hotel, airport, etc.).  The time and cost to travel
from the source to the first POI and from the last POI to the
destination has to be also taken into account.

\begin{figure}[t]
	\centering
    \includegraphics[width=0.75\columnwidth]{plots/updatedExample.png}
	\caption{Example of an optimal itinerary (solid edges represent walking and dashed edges represent taxi; edge labels depict \{travel cost, travel time\} while vertex labels depict \{visit cost, visit  time, utility\}; S and D are source and destination)}
	\label{fig:example_graph}
\end{figure}

\begin{table}[t]
	\centering
	\resizebox{0.85\columnwidth}{!}
	{
		\begin{tabular}{c l rrr}
			\toprule
			\textbf{ID} & \textbf{Category} & \textbf{Visit Time} & \textbf{Utility} & \textbf{Visit Cost} \\
			\midrule
			%S & Source      & 0  & 0   & 0    \\
			1 & Park        & 90 & 886 & 1320 \\
			2 & Park        & 30 & 253 & 440  \\
			3 & Park        & 45 & 599 & 275  \\
			4 & Museum      & 60 & 512 & 4730 \\
			5 & Museum      & 30 & 315 & 330  \\
			6 & Museum      & 60 & 660 & 825  \\
			%D & Destination & 0  & 0   & 0    \\
			\bottomrule
		\end{tabular}
	}
	\caption{Details of Points of Interests (POIs)}
	\label{tab:example_poi}
\end{table}

An itinerary $I$ of $n$ POIs is, thus, represented as a sequence
%
\begin{align}
	I = [ S, v_1, \dots, v_n, D ]
\end{align}
%
The total time spent for itinerary $I$ is
%
\begin{align}
	t(I) = t_{S,v_1} + t_{v_1} + t_{v_1,v_2} + t_{v_2} + \dots + t_{v_{n-1,v_n}} + t_{v_n} + t_{v_n,D}
\end{align}
%
which takes into account both the travel time and visiting time.  Similarly,
the total cost of itinerary $I$ is 
%
\begin{align}
	c(I) = c_{S,v_1} + c_{v_1} + c_{v_1,v_2} + c_{v_2} + \dots + c_{v_{n-1,v_n}} + c_{v_n} + c_{v_n,D}
\end{align}
%
which again takes into account both the travel cost and visiting cost.  The
utility obtained from $I$ is
%
\begin{align}
	U(I) = U_{v_1} + U_{v_2} + \dots + U_{v_n}
\end{align}
%
Given a time budget $T$ and a cost budget $C$, the TRIP problem is to find the
itinerary $I$ that maximizes $U(I)$ subject to the budget constraints:
%
\begin{align}
	& \arg\max_I U(I) \\
	\text{subject to } & t(I) \leq T \text{ and } c(I) \leq C
\end{align}

The traveler may put additional constraints on the itinerary such as
a particular POI must be visited (e.g., Eiffel Tower in Paris).  She
may also put restrictions in the form that not more than $k$ POIs of
the same category may be included and/or at least one POI of a particular
category must be included, etc.  She may also put in ordering
constraints, such as a temple must be visited before a museum, etc.  The feasible itinerary $I$ must satisfy all these
constraints.

\ignore{

In this work, we introduce the \emph{Trip Itinerary Planning (TRIP) Problem},
which is formulated as follows. The city is represented as a multimodal graph
of points of interest (POIs), with every pair of POIs connected by two edges
corresponding to walking and taxi travel. The taxi speed is $s_t$ and walking
speed is $s_w$. Using these, the time to travel from one POI to another is
calculated. Similarly, there is a cost per distance unit $c_t$ for taxi (the
corresponding cost for walking is $c_w = 0$), using which the cost of traveling
from one POI to another is computed. Every POI has a \emph{utility score},
which is its relative attractiveness or significance to the tourist. Given a
cost and a time budget, the basic problem is to \emph{find an itinerary}, i.e.,
a sequence of POIs, that \emph{maximizes the overall utility} obtained from the
chosen POIs. The core of the system is formulated as a Mixed-Integer Linear
Programming (MILP) problem, solved using the Gurobi Optimizer\footnote{\ab{cite
the tool}}, a powerful tool for mathematical optimization.
%The goal of the MILP model is to maximize the total utility obtained from the
%chosen POIs in the resulting itinerary, subject to a rich set of real-world
%constraints.

The itinerary is a linear, day-wise sequence of POIs that the tourist traverses
in the city over multiple days, while adhering to daily time constraints and
overall cost constraint. Let $V$ denote the set of POIs in the city. Each POI
has its associated opening and closing time, i.e., the time window in which it
can be explored by the tourists. Also, each POI has a weekly schedule and
specific operating days on which it can be visited.

\ab{This section is too much to-and-fro; there is not much of a flow -- we need to discuss}

Each POI \( v_i \in V \) has the following features:

\begin{itemize}
    \item \( U(v_i) \): Utility Score associated with each POI
    \item \( ot(v_i) \): Opening time of POI
    \item \( ct(v_i) \): Closing time of POI
    \item \( t(v_i) \): Average visit time spent by tourists on the POI
    \item \( c(v_i) \): Entrance fee for the POI.
\end{itemize}

\noindent \textbf{Multi-modal Graph Representation:}
This problem is modeled using a multi-modal graph \( G = (V, E) \) where:

\begin{itemize}
    \item \( V = \{v_1, ..., v_N\} \) denotes the set of all Points of interest of the city
    \item \( E \) contains two distinct edges between every pair of POIs \( (v_i, v_j) \), corresponding the two modes of travel:
    \begin{itemize}
        \item \textbf{Walking Edge} \( e^{w}_{i,j} \) is characterized by:
        \begin{itemize}
            \item \( t^{w}_{i,j} \): Time required to walk from \( v_i \) to \( v_j \).
            \item \( c^{w}_{i,j} \): Cost associated with walking (usually zero).
        \end{itemize}
        \item \textbf{Taxi Edge} \( e^{t}_{i,j} \) is characterized by:
        \begin{itemize}
            \item \( t^{t}_{i,j} \): Time required to travel by taxi.
            \item \( c^{t}_{i,j} \): Cost of travel by taxi.
        \end{itemize}
    \end{itemize}
    \item \textit{days} is the set of all days.
\end{itemize}

\noindent \textbf{Objective Function}\\
To create multi-day itinerary, the optimization ensures a balanced distribution across all days. The objective is to \textbf{maximize the total utility score across all days}.

The decision variable \( y_{i,d} \in \{0,1\} \) indicates whether POI \( v_i \) is selected on day \( d \). The objective function is:
\begin{align}
U(I) &= \sum_{d \in \text{days}} \sum_{i = 1}^{N} U(v_i) \cdot y_{i,d} \label{eq:multi_day_binary}
\end{align}

This equation computes the total utility $U(I)$ of the multi-day itinerary by summing the utilities of all selected POIs across all days.

The basic constraints of the system are described below:

\begin{itemize}

\item \textbf{Time Budget Constraint}\\
This constraint ensures that the total time taken i.e. the sum of visit and travel times is under the user specified time budget for each day.

\begin{align}
\label{mul_day_9}
    & \sum_{i \ne j} t^{w}_{i,j} \cdot w_{i,j,d}
    + \sum_{i \ne j} t^{t}_{i,j} \cdot x_{i,j,d}
    + \sum_{i \in V} t(v_i) \cdot y_{i,d} \leq H
\end{align}

where H is the time budget specified by user. The user can specify different time budgets for the first day, last day, and intermediate days to account for flight or train arrival and departure timings.
\\[1ex]
\item \textbf{Cost Budget Constraint} \\
 This constraint ensures that the total cost of the trip, including both taxi travel and POI entry fees, remains within the cost budget specified by the user.

\begin{align}
\label{mul_day_25}
\sum_{d \in \text{days}} \sum_{(i,j) \in I} \left(c^{w}_{i,j} \cdot w_{i,j} + c^{t}_{i,j} \cdot x_{i,j} \right) + \sum_{d \in \text{days}} \sum_{i \in I} c(v_i) \leq B 
\end{align}
where B is the cost budget for whole trip.
\\[1ex]
\item \textbf{Opening and Closing Time}\\
    This constraint makes sure that each POI is visited correctly in its operating hours:
    \begin{equation}
    \label{mul_day_29}
    s_{i,d} \geq \text{ot}_i \quad \forall i \in \text{{poi\_ids}}, \forall d \in \text{days} \\
    \end{equation}
    \begin{equation}
    \label{mul_day_30}
        s_{i,d} \leq \text{ct}_i - t(v_i) \quad \forall i \in \text{{poi\_ids}}, \forall d \in \text{days}
    \end{equation}
    \noindent
    where \( s_{i,d} \) is the arrival time at POI \( i \) on day \( d \)
\\[1ex]
\item \textbf{Opening and Closing Day}\\
    This constraint checks the availability of each POI on the day selected by the user. If the POI is not open for public during that day, then this constraint effectively uses $y[i, d] = 0$, to deliberately not include that POI in that day's itinerary.
    \begin{align}
    \label{mul_day_31}
 \quad y_{i,d} = 0 \quad  &\forall i \in \text{POI\_IDs}, \nonumber \\
        &\forall d \in \text{days, } \texttt{day\_availability}[\texttt{trip\_weekdays}[d]][i] = 0 \nonumber
\end{align}
    % \[\quad \forall i \in \text{{poi\_ids}}, \forall d \in \text{Days},\]
    \noindent
    where:
    \begin{itemize}
        \item \textit{days} represent the set of weekdays.
        \item \( \texttt{trip\_weekdays}[d] \) denotes the weekday corresponding to day \( d \).
        \item \( \texttt{day\_availability}[weekday][i] \) is 1 if POI \( i \) is open on that weekday, and 0 if it is closed.
    \end{itemize}
\end{itemize}

\textbf{Example}

}

Consider an example of a city that has 6 POIs.
Table~\ref{tab:example_poi} shows the various time and cost values for
the POIs while Table~\ref{tab:example_walk} and
Table~\ref{tab:example_taxi} shows the travel times among the POIs and
the source and destination marked by $S$ and $D$ respectively for
walking and taxi respectively.

\begin{table}[t]
	\centering
	\resizebox{\columnwidth}{!}
	{
		\begin{tabular}{c|cccccccc}
			\toprule
			\textbf{From$\backslash$To} & \textbf{S} & \textbf{1} & \textbf{2} & \textbf{3} & \textbf{4} & \textbf{5} & \textbf{6} & \textbf{D} \\
			\midrule
			\textbf{S} & --    & 28.7  & 49.9  & 55.5  & 68.9  & 126.6 & 117.2 & 102.9 \\
			\textbf{1} & 28.7  & --    & 3.3   & 3.2   & 4.3   & 10.1  & 9.3   & 78.2  \\
			\textbf{2} & 49.9  & 3.3   & --    & 5.7   & 2.7   & 8.0   & 6.9   & 94.3  \\
			\textbf{3} & 55.5  & 3.2   & 5.7   & --    & 5.0   & 9.9   & 9.6   & 47.4  \\
			\textbf{4} & 68.9  & 4.3   & 2.7   & 5.0   & --    & 5.8   & 5.0   & 74.7  \\
			\textbf{5} & 126.6 & 10.1  & 8.0   & 9.9   & 5.8   & --    & 1.5   & 98.5  \\
			\textbf{6} & 117.2 & 9.3   & 6.9   & 9.6   & 5.0   & 1.5   & --    & 102.4 \\
			\textbf{D} & 102.9 & 78.2  & 94.3  & 47.4  & 74.7  & 98.5  & 102.4 & --    \\
			\bottomrule
		\end{tabular}
	}
	\caption{Walking travel time matrix}
	\label{tab:example_walk}
\end{table}

\begin{table}[t]
	\centering
	\resizebox{0.97\columnwidth}{!}
	{
		\begin{tabular}{c|cccccccc}
			\toprule
			\textbf{From$\backslash$To} & \textbf{S} & \textbf{1} & \textbf{2} & \textbf{3} & \textbf{4} & \textbf{5} & \textbf{6} & \textbf{D} \\
			\midrule
			\textbf{S} & --    & 3.8  & 6.7  & 7.4  & 9.2  & 16.9 & 15.6 & 13.7 \\
			\textbf{1} & 3.8   & --   & 0.4  & 0.4  & 0.6  & 1.3  & 1.2  & 10.4 \\
			\textbf{2} & 6.7   & 0.4  & --   & 0.8  & 0.4  & 1.1  & 0.9  & 12.6 \\
			\textbf{3} & 7.4   & 0.4  & 0.8  & --   & 0.7  & 1.3  & 1.3  & 6.3  \\
			\textbf{4} & 9.2   & 0.6  & 0.4  & 0.7  & --   & 0.8  & 0.7  & 10.0 \\
			\textbf{5} & 16.9  & 1.3  & 1.1  & 1.3  & 0.8  & --   & 0.2  & 13.1 \\
			\textbf{6} & 15.6  & 1.2  & 0.9  & 1.3  & 0.7  & 0.2  & --   & 13.7 \\
			\textbf{D} & 13.7  & 10.4 & 12.6 & 6.3  & 10.0 & 13.1 & 13.7 & --   \\
			\bottomrule
		\end{tabular}
	}
	\caption{Taxi travel time matrix}
	\label{tab:example_taxi}
\end{table}

Suppose a traveler has a time and cost budget of $360$ minutes and $3500$ units respectively.
Additionally, she puts in the contraints that (1)~POIs 1 and 2 must be visited, (2)~POI 2 must be visited before POI 1, and (3)~POI 3, if visited, must be before both POI 2 and POI 5 (if visited).
Further, she must visit at least 2 POIs of the category Park, and will not visit more than 2 POIs of the category Museum.

Respecting all these constraints, the optimal itinerary is $[S, 2, 5, 6, 1, D]$, as shown in Figure~\ref{fig:example_graph}.  Note that the
POIs 3 and 4 were not included.  The total utility obtained from the
itinerary is $2114$ units, and the time and cost spent 
are, respectively, $341.9$ minutes and $3260$ units.

\ignore{

\textbf{Constraints applied:}
\begin{enumerate}[label=\textbf{\arabic*.}]
    \item \textbf{Time budget:} 360 minutes (5 hours)
    \item \textbf{Cost budget:} 3500 units
    \item \textbf{Must-see POIs:} POIs 1 and 2 must be included in the itinerary
    \item \textbf{Ordering constraints:} (Applied if both POIs are included in the itinerary)
    \begin{itemize}
        \item POI 3 must be visited before POI 2
        \item POI 2 must be visited before POI 1
        \item POI 3 must be visited before POI 5
    \end{itemize}
    \item \textbf{Category constraints:}
    \begin{itemize}
        \item At least 2 POIs from the \textit{Park} category
        \item At most 2 POIs from the \textit{Museum} category
    \end{itemize}
    \item \textbf{Modes of travel allowed:} Taxi and walking
\end{enumerate}
}


\section{The \trip Solution Framework}

\begin{align}
    & \textbf{maximize} \sum_{d \in \text{days}} \sum_{i = 1}^{N} U(v_i) \cdot y_{i,d} \hspace{3.4cm} \ref{eq:multi_day_binary} \notag\\
    & \textbf{subject to:} \notag \\ 
    & \sum_{d \in \text{days}} y_{i,d} \leq 1 \quad \forall i \in \text{poi\_ids} \setminus \{\mathit{hotel\_id}\}\\
    & \text{day\_visit}_i = \sum_{d \in \text{days}} d \cdot y_{i,d}\quad \forall i \in \text{poi\_ids} \\
    & z_{i,j,d} = w_{i,j,d} + x_{i,j,d} \quad\forall i, j \in \text{poi\_ids},\, i \ne j,\; \forall d \in \text{days} \\
    & z_{i,j,d} \leq 1 \quad \forall i, j \in \text{poi\_ids},\, i \ne j,\; \forall d \in \text{days} \\
    &  z_{i,j,d} \leq y_{i,d} \quad \forall d \in \text{days},\quad \forall i, j \in \text{poi\_ids},\; i \ne j\  \\
    &  z_{i,j,d} \leq y_{j,d}, \quad \forall d \in \text{days},\quad \forall i, j \in \text{poi\_ids},\; i \ne j\ \\   
    & s_{i,d} \leq day\_start\_time + day\_budget + (1 - y_{i,d}) \cdot M \notag\\
    &\forall i \in \text{poi\_ids}, \forall d \in \text{days}\  \\
    & \textbf{1. First Day } (d = d_1): \notag \\
    & \sum_{\substack{j \in \text{POIs} \\ j \neq \text{starting-poi}}} z_{\text{source-poi},j,d_1} = 1 \\
    & \sum_{\substack{i \in \text{POIs} \\ i \neq \text{hotel}}} z_{i,\text{hotel},d_1} = 1  \\
    & \sum_{\substack{j \in \text{POIs} \\ j \neq \text{hotel}}} z_{\text{hotel},j,d_1} = 0  \\
    & \textbf{2. Last Day } (d = d_T): \notag \\
    & \sum_{\substack{j \in \text{POIs} \\ j \neq \text{hotel}}} z_{\text{hotel},j,d_T} = 1 \\
    & \sum_{\substack{i \in \text{POIs} \\ i \neq \text{ending-poi}}} z_{i,\text{ending-poi},d_T} = 1 \\
    & \sum_{\substack{j \in \text{POIs} \\ j \neq \text{ending-poi}}} z_{\text{ending-poi},j,d_T} = 0 \\
    & \textbf{3. Intermediate Days } (d \in \text{days} \setminus \{d_1, d_T\}): \notag \\[0.2cm]
    & \sum_{\substack{j \in \text{POIs} \\ j \neq \text{hotel}}} z_{\text{hotel},j,d} = 1 \quad \forall d \in \text{days} \setminus \{d_1, d_T\} \\
    & \sum_{\substack{i \in \text{POIs} \\ i \neq \text{hotel}}} z_{i,\text{hotel},d} = 1 \quad \forall d \in \text{days} \setminus \{d_1, d_T\} \\
    & \sum_{\substack{i \in \text{POIs} \\ i \neq k}} z_{i,k,d} = y_{k,d}
    \quad \forall d \in \text{days},\forall k \in \text{POIs} \setminus \{\text{start}, \text{end}, \text{hotel}\}\\
    & \sum_{\substack{j \in \text{POIs} \\ j \neq k}} z_{k,j,d} = y_{k,d}, \quad \forall d \in \text{days}, \forall k \in \text{POIs} \setminus \{\text{start}, \text{end}, \text{hotel}\}
     \end{align}
\begin{align}
    & \sum_{i \ne j} t^{w}_{i,j} \cdot w_{i,j,d} + \sum_{i \ne j} t^{t}_{i,j} \cdot x_{i,j,d} + \sum_{i \in V} \text{t}(v_i) \cdot y_{i,d} \nonumber \leq H \hspace{0.9cm} \ref{mul_day_9}\\
    & \sum_{d \in \text{days}} \sum_{(i,j) \in I} c^{w}_{i,j} \cdot w_{i,j} + c^{t}_{i,j} \cdot x_{i,j} + \sum_{d \in \text{days}} \sum_{i \in I} c(v_i) \leq B \hspace{0.3cm} \ref{mul_day_25} \notag\\
    & s_{i,d} \geq \left( s_{j,d} + t(v_j) + t^{w}_{j,i,d} \cdot w_{j,i,d} + t^{x}_{j,i,d} \cdot x_{j,i,d} \right) \cdot z_{j,i,d} \notag\\
    &- M_{ji} \cdot (1 - z_{j,i,d}) \quad\forall i \neq j,\; \forall d \in \text{days} \\
    & ot(v_i) \leq s_{i,d} \leq ct(v_i) - t(v_i) \quad \forall v_i \in V, \forall d \in \text{days} \hspace{1cm}\ref{mul_day_30} \notag\\
    & y_{i,d} = 0 \quad \forall i \in \text{POIs}, \forall d \in \text{days} \notag \\& \text{ if } \texttt{day\_availability}[trip\_weekdays][d]][i] = 0 \hspace{1cm}\ref{mul_day_31} \notag\\
    & s_{start\_node,d} = \text{day\_start\_time} \quad \forall d \in \text{days}
\end{align}

The objective function (1) maximises the overall utility of the trip. Constraint (7) ensures that each POI is visited atmost once in the whole trip. Constraint (8) assigns a day to each POI on which it is visited during the itinerary. Constraints (9) and (10) ensure that only one mode out of Taxi and Walk is selected. Constraints (11) and (12) establish a logical connection between edges and vertices by preventing isolated edges and vertices. Constraint (13) puts an upper bound on the arrival time of each POI. Constraints (14),(15) and (16) make sure that the itinerary of first day is correctly modelled from source POI to hotel POI. Similarly, Constraints (17),(18) and (19) make sure that the itinerary of last day is correctly modelled from hotel POI to destination POI, and Constraints (20) and (21) help in modelling the itinerary of intermediate days from hotel to hotel correctly. Constraints (22) and (23) are connectivity constraints, which impose a restriction that if an edge enters a vertex, another edge must leave that vertex. Constraints (2),(3),(5),(6) are time budget, cost budget, opening closing time and opening closing day constraints respectively that are already explained in previous section. Constraint (24) ensures that if an edge goes from POI j to POI i, then the arrival time of i is greater than or equal to the sum of visit time of POI j and travel time between j to i. M is a large constant which relaxes the constraint when both of them aren't part of itinerary. Constraint (25) ensures that each day's itinerary starts on correct time as stated by the tourist.



\section{Extensions}

\textbf{Dynamic Re-routing}\\
The real tourist behavior and weather conditions, are dynamic and unpredictable in nature. They can be influenced by unforeseen delays, detours, longer-than-anticipated visits, or spontaneous user preferences, which can significantly impact the feasibility and correctness of an advanced preplanned itinerary. To close the gap between theory and practice, a dynamic approach is required.

In contrast with the static approach used in \cite{taylor2018tour}, where rigid, precomputed travel times and POI visit times, we have taken real-time travel times using Google-Maps API (Routes API). 

On the top of suggested times of visit and travel, user can enter the actual time they have taken to visit a POI and the travel time they took to reach the current POI.

With every visit to a POI, the system replans the schedule based on the remaining time and cost budgets. The POIs already visited are not considered for further itineraries. This optimizes the rest of the day based on current progress and not stale assumptions.

For the dynamic feature, a record of visited POIs was maintained. On visiting a POI, it was eliminated from the list to be taken into account in future itinerary calculations with the new remaining time and cost budgets.

\begin{figure}[H]
\textbf{Implementation of Dynamic Approach}
\centering
\includegraphics[width=0.5\textwidth]{binary dynamic flowchart.png}
\caption{Implementation of Dynamic Approach}
\label{fig:flowchart_dynamic}
\end{figure}

\noindent \textbf{Fractional Visits to POIs}\\
This feature enables partial visiting of Points of Interest (POIs), thus enhancing the flexibility and potential utility of the proposed itineraries. In comparison to the Binary approach used before which required the POI to be completely visited in order to collect its utility, the Fractional version built here grants proportional utility based on the proportion visited of the POI—improving time and cost budget efficiency.

\textbf{Fractional POI Variable (ppoi[i])}
\begin{itemize}
    \item A real variable $\text{ppoi}[i] \in [0, 1]$ is defined for every POI i to reflect the portion of the POI's total visit time covered by the itinerary.

    \item A threshold of 0.5 (50\%) is enforced: POIs must be visited for at least half of their average duration to be included. If ppoi[i] < 0.5, it is effectively set to 0 and the POI is excluded from the plan.
\end{itemize}

\textbf{Utility Calculation Variants}
\label{Utility_Calculation}

Two distinct methods are used to compute the utility in the fractional setting:

\begin{itemize}
\item {Continuous Linear Utility Model}

In this version, the utility granted is directly proportional to the fraction of the POI visited.

If a POI has utility $U$ and is visited for $p$ fraction of time (where $p \in [0.5, 1]$), the utility granted is $U \times p$.

\item{Slab-Based Utility Model}
\begin{itemize}[noitemsep, topsep=0pt]
    \item Fraction $\in$ [0.5, 0.6) $\rightarrow$ 50\% utility
    \item Fraction $\in$ [0.6, 0.7) $\rightarrow$ 60\% utility
    \item Fraction $\in$ [0.7, 0.8) $\rightarrow$ 70\% utility
    \item Fraction $\in$ [0.8, 0.9) $\rightarrow$ 80\% utility
    \item Fraction $\in$ [0.9, 1.0) $\rightarrow$ 90\% utility
    \item Fraction = 1.0 $\rightarrow$ 100\% utility
\end{itemize}

It is important to note that we provide 100\% utility if and only if POI is visited completely. The utility granted coincides with the lower bound because we do not want a situation like this to occur where even if we suggest that tourist visit 50\% POI and they get any utility greater than 50\% because that would scale up the actual utility as compared to the continuous linear function.
\end{itemize}

To enable fractional visits, the constraints were modified to make the visit duration at one POI proportional to a continuous variable \( ppoi_i \in [0,1] \), that represents the relative fraction of the overall visit duration spent at the POI \( v_i \). Accordingly, in all the constraints, the original visit times are multiplied by this fractional variable \( ppoi_i \), ensuring the constraints accurately reflect partial visits.

When the utility function is a continuous linear function of visit time, the utility score obtained from a POI is also scaled proportionally, i.e., \( \text{utility}_i = U(v_i) \cdot ppoi_i \), where \( U(v_i) \) is the full utility for a complete visit to POI \( v_i \).

For the slab-based utility variant, a continuous variable \( \text{effective\_utility}_i \) is used to capture the actual utility awarded based on the fraction of time spent at each POI. The utility is determined using discrete slab multipliers based on the visit duration. Each POI is assigned to at most one slab, and the utility is calculated as:

\begin{align}
\label{effective_utility}
\text{effective\_utility}_i &= U(v_i) \cdot \big(0.5 \cdot s_1[i] + 0.6 \cdot s_2[i] + 0.7 \cdot s_3[i] \notag \\
&\quad + 0.8 \cdot s_4[i] + 0.9 \cdot s_5[i] + 1.0 \cdot s_6[i] \big)
\end{align}

The optimization objective is to maximize the total utility accumulated across all POIs:

\begin{align}
\label{objective_fun_slabs}
U(I) = \sum_{i=1}^N \text{effective\_utility}_i
\end{align}

To ensure only one slab level is chosen per POI, the following constraint is imposed:

\begin{align}
\sum_{k=1}^{6} s_k[i] \leq 1, \quad \forall i \in \{1, \dots, N\}
\end{align}


\section{Experiments}
\label{sec:experiments}

In this section, we describe the empirical study of our \trip solver for the itinerary planning problem.

\subsection{Dataset Description}
\label{sec:dataset}

In order to test our itinerary planning system, we used curated datasets of Points of Interest (POIs) for six large cities: Budapest, Delhi, Edinburgh, Glasgow, Osaka, and Vienna. 
%\ari{Americans will be very unhappy.} \ab{Ha ha ha!}
The dataset utilized for our work is derived from the \emph{Yahoo Flickr Creative Commons 100 Million Dataset (YFCC100M)}~\cite{taylor2018tour} containing over 100 million images, of which 69 million are annotated and 48 million are geotagged.
%\ab{69+48>100.} \ari{Some of the images are common, i.e., both annotated and geotagged.}
%
The authors of~\cite{taylor2018tour} considered a set of popular POIs for the cities mentioned earlier using resources such as \emph{Wikipedia}. Then, geotagged photos in \emph{Flickr} were matched with these POIs using spatial proximity. The relative popularity of a POI, used as the measure for its \textbf{utility}, was estimated based on the number of photos associated with each POI. The Flickr User-POI Visits dataset, curated in this manner by~\cite{limkwanhuiDataCode}, is used in our work.
This provides a practical way to understand how tourists behave by using publicly shared photos as a proxy for how much interest people have in a particular place.

Table~\ref{tab:original} contains the original dataset fields. With an aim to make our system more realistic and practical to use, we augmented the fields by adding some more features, as listed in Table~\ref{tab:additional}.
These additional information were collected by either scraping or quoting official tourist boards, city tourism websites, and trustworthy travel websites.

\begin{table}[t]
\begin{tabular}{l l}%p{4cm}}
\toprule
\textbf{Field Name} & \textbf{Description} \\
\midrule
\texttt{ID} & Unique ID of the POI\\%assigned to each POI. \\
%\hline
\texttt{Name} & Name of the POI\\%, e.g., ``Red Fort'', ``Osaka Castle''. \\
%\hline
\texttt{Location} & Location of the POI in terms of latitude and longitude \\
%\texttt{Latitude} & Latitude\\%coordinate of the POI. \\
%\hline
%\texttt{Longitude} & Longitude\\%coordinate of the POI. \\
%\hline
\texttt{Category} & Theme or Type of the POI\\% classification of the POI, such as \textit{amusement}, \textit{historical}, \textit{museum}, \textit{shopping}, \textit{park}, etc. \\
%\hline
\texttt{Utility} & Numerical value representing estimated utility of the POI\\% or attractiveness of the POI, derived based on its popularity (photo frequency). \\
%\hline
\texttt{Distance} & Distance to other POIs\\%. This is used in the travel time estimation between locations, with the walking speed $v_w$ and taxi speed $v_t$. \\
\bottomrule
\end{tabular}
\caption{Original dataset fields}
\label{tab:original}
\end{table}

%\subsection{Experimental Setup}

%With an aim to make our system more realistic and practical to use, we supplemented the dataset manually with real-life operating limitations and data that we gathered from \textbf{official tourist websites} and verified online portals. The added fields are:\pri{This section also needs modification}

\begin{table}[t]
\centering
\begin{tabular}{l l}
\toprule
\textbf{Field Name} & \textbf{Description} \\
\midrule
\texttt{Visit Cost} & Visit cost of the POI including entrance fee, etc.\\%Entrance fee or ticket price associated with the POI, in INR. \\
%\midrule
\texttt{Visit Time} & Average time spent by tourists in the POI\\%duration (in minutes) tourists typically spend at the POI. \\
%\midrule
\texttt{Opening Time} & Time of the day when the POI opens\\%The time at which the POI opens for visitors, stored in \texttt{HH:MM:SS} format. \\
%\midrule
\texttt{Closing Time} & Time of the day when the POI closes\\%The time at which the POI closes for visitors, stored similarly. \\
%\midrule
\texttt{Days of Week} & Days when the POI is open\\%Seven binary columns (\texttt{Monday}, \texttt{Tuesday}, ..., \texttt{Sunday}). A value of 1 indicates the POI is open on that day; 0 indicates it is closed. \\
\texttt{Travel Time} & Travel time to other POIs for each mode of transport \\
\texttt{Travel Cost} & Travel cost to other POIs for each mode of transport \\
\bottomrule
\end{tabular}
\caption{Additional features added to the dataset}
\label{tab:additional}
\end{table}

\ignore{

These manually extracted features add a \textbf{temporal and availability aspect} to the optimization problem, allowing more realistic and accurate itinerary generation. For example, POIs closed on the chosen day are excluded from the planning automatically.

}

\subsection{Configuration}

The experiments were conducted on a high-performance Linux server equipped with dual Intel(R) Xeon(R) E5-2697 v3 CPUs, each running at 2.60\,GHz, providing a total of 56 logical processors (28 physical cores per socket with hyper-threading enabled) and 503\,GB of RAM.
%The system had a substantial 503\,GB of RAM and supported 64-bit operations, offering a powerful and memory-rich environment for executing complex computations.
The code was implemented in Python and mixed integer linear programming was done in Gurobi.
%The implementation was carried out in Python, utilizing its extensive ecosystem for data manipulation, interaction, and visualization. Optimization tasks were handled by the Gurobi Optimizer, which solved the Mixed Integer Linear Programming (MILP) models required to generate optimal multi-day travel itineraries under user-defined constraints and real-world constraints.

\subsection{Baseline Itinerary Planner}

To evaluate the performance of the \trip solution, we identified the works \cite{bolzoni2014efficient,taylor2018tour} and \cite{chen2014automatic,vanzelst2016itinerary} as potential baselines for single day itinerary planning and multi-day itinerary planning, respectively. This choice was based on the features of the solutions, as listed in Table~\ref{tab:otherworks}. Although we requested the authors of the respective works to share their implementation codes we did not receive a suitable response. Moreover, while the works \cite{bolzoni2014efficient,taylor2018tour}  consider utility optimization as their objective, the works \cite{chen2014automatic,vanzelst2016itinerary} did not. In addition, the implementation details stated in \cite{bolzoni2014efficient} were insufficient to reproduce their solution.  We, thus, did a best-effort implementation of~\cite{taylor2018tour}. While it too lacked a code repository or implementation details like the POI visiting durations, we were able to at least replicate the key features and constraints described. 
Hence, we consider \cite{taylor2018tour} to be the  \emph{baseline} for evaluating single day as well as multi-day itinerary planning.

\ab{re-write the rest of this section}

The baseline model incorporates only fundamental constraints, including the time budget constraint, connectivity requirements, a restriction preventing a direct path between the start and the end POIs, and sub-tour elimination constraints which the authors mentioned in their paper. In our implementation, the sub-tour elimination is effectively enforced using arrival-time-based constraints. We were able to seamlessly adapt our MILP-based framework to replicate this baseline behavior, effectively converting our advanced planner into a simplified version matching the baseline model's structure.

However, direct performance comparison between our model and the baseline was not feasible due to a key limitation in the common dataset used by both our system and Taylor and Lim \cite{taylor2018tour} --namely, the absence of standardized POI visiting durations. Despite this, we were able to re-implement the baseline model in its entirety in terms of constraints as well as utility structure, which provided a good basis for qualitative analysis. 

Another limitation of the itinerary planner described in Taylor and Lim \cite{taylor2018tour} was that it did not consider any cost budget during itinerary planning. To address this, we incorporated cost constraints into our model. However, to ensure a fair comparison with their approach, we used a very high cost budget in our experiments related to the Baseline Comparison, so that it would not influence the itinerary planning outcome.
\subsection{Variants of \trip}

We evaluate the performance of the \trip solution by considering all the utility variants (binary, slab, continuous) as well as the transportation mode variants (walking, taxi, hybrid). We, thus, have a total of $3 \times 3 = 9$ \trip variants.

\ignore{

as listed in Table~\ref{tab:trip_variants}, that are based on the transportation mode and the chosen utility variant. 
\begin{table}[th]
\centering
\begin{tabular}{|l|l|l|}
\hline
\textbf{TRIP VARIANT} & \textbf{Transportation Mode} & \textbf{Utility Variant} \\
\hline
TRIP\_W\_B & Walk & Binary \\
TRIP\_T\_B & Taxi & Binary \\
TRIP\_H\_B & Hybrid -- Walk + Taxi & Binary \\
TRIP\_W\_C & Walk & Fractional -- CLF \\
TRIP\_T\_C & Taxi & Fractional -- CLF \\
TRIP\_H\_C & Hybrid -- Walk + Taxi & Fractional -- CLF \\
TRIP\_W\_S & Walk & Fractional -- Slabs \\
TRIP\_T\_S & Taxi & Fractional -- Slabs \\
TRIP\_H\_S & Hybrid -- Walk + Taxi & Fractional -- Slabs \\
\hline
\end{tabular}
\caption{\trip Variants by Transportation Mode and Utility Variant \nl{CLF abbreviation}}
\label{tab:trip_variants}
\end{table}

}

\subsection{Performance Metrics}
%\subsection{Input Parameters and Performance Metrics}

\ignore{

\ab{isn't this repetitive? we have already said this so many times -- in intro, solution, etc.!}

\noindent \textbf{User Inputs}\\
Our trip planning system is capable of accepting a wide range of user inputs to support personalized and realistic trip planning. They are:

\begin{itemize}
    \item \textbf{City Choice:} The city would be selected by the user from options.
    
    \item \textbf{Trip Day:} The actual day on which the trip is to take place.

    \item \textbf{Start and End Points:} Latitude and longitude coordinates identifying the point where the day's journey begins and ends.
    
    \item \textbf{User Preferences:}
    \begin{itemize}
        \item \textbf{Category Constraints:} Limit on the category of POIs to be addressed, i.e., museum, market, park, etc.
        \item \textbf{Must-See and Must-Exclude POIs:} Individual POIs that the user wants to add or remove from the itinerary.
        \item \textbf{Ordering Constraints:} Ordering constraints with respect to POIs, indicating visit priority.
    \end{itemize}
    
	\item \textbf{Utility Variant:} User can choose from 3 different utility variants which are Binary, Continuous and Slabs. \ab{Add the utility variants}
    \item \textbf{Time Budget:} The time (minutes or hours) the user is willing to spend on the trip.
    \item \textbf{Cost Budget:} The maximum cost which the user will incur on travel expenditure (e.g., taxi fares) and entry fees of POIs.
    
    \item \textbf{Dynamic Variant Specific Inputs:}
    \begin{itemize}
        \item \textbf{Actual Visitation Time per POI:} Used to update the remaining itinerary dynamically during execution.
        \item \textbf{Actual Travel Time between POIs:} Used to dynamically recalculate transitions and reschedule the visits accordingly.
    \end{itemize}
    
    \item \textbf{Multi-day Trip Specific Inputs:}
    \begin{itemize}
        \item \textbf{Number of Days:} Total number of days in the trip.
        \item \textbf{Start, End, and Hotel Coordinates:} Coordinates specifying the start, end, and overnight locations for each day.
        \item \textbf{Start Time \& Time Budget per Day:} Daily starting time and allowed time budget for first, last and intermediate days.
    \end{itemize}
\end{itemize}

\noindent\textbf{Performance Metrics}\\
We evaluate the quality and efficiency of the generated itineraries using the following performance metrics:

\begin{itemize}
    \item \textbf{Utility:} The total utility obtained from the selected POIs in the itinerary, reflecting the quality and effectiveness of the itinerary.
    
    \item \textbf{Running Time:} The time (in seconds) required by the system to compute the itinerary. It is an indicator of the system's computational efficiency.

    \pri{Need a discussion on these}
    \item \textbf{Time Utilization:} The percentage of the user-provided time budget that is actually used in the itinerary:
    \[
    \text{Time Utilization \%} = \frac{\text{Total time used}}{\text{Time budget}} \times 100
    \]

     \item \textbf{Cost Utilization:} The percentage of the user-provided cost budget that is actually used in the itinerary:
    \[
    \text{Cost Utilization \%} = \frac{\text{Total cost used}}{\text{Cost budget}} \times 100
    \]
    
    
    
    \item \textbf{Fraction of Time Spent in Travel and Visits:}
    \begin{itemize}
        \item \textbf{Travel Time Fraction:} Ratio of time spent traveling between POIs to the total utilized time.
        \item \textbf{Visit Time Fraction:} Ratio of time spent visiting POIs to the total utilized time.
    \end{itemize}

    \item \textbf{Fraction of Cost Spent in Travel and Visits:}
    \begin{itemize}
        \item \textbf{Travel Cost Fraction:} Ratio of cost spent traveling between POIs to the total utilized cost.
        \item \textbf{Visit Cost Fraction:} Ratio of cost spent in entrance fee of POIs to the total utilized cost.
    \end{itemize}
\end{itemize}

\ab{Add Cost Utilization}

These metrics serve as the basis for the comparative analysis and graphical illustrations presented in the subsequent section.

}

We evaluate the performance of the baseline and the \trip variants on mainly \emph{utility}, since that is what the objective of the optimization function is.
In addition, however, we also measure the following parameters:

\begin{enumerate}
    \item \emph{Running time} of the solver, to determine its practicality
    \item \emph{Time and Cost spent} as a ratio of the total time and cost budget, to understand the utilization efficiency
    \item \emph{Time and Cost profile} of how time and cost are spent for travel vis-a-vis POI visit
\end{enumerate}

\begin{figure}[t]
\centering
\includegraphics[width=\columnwidth]{plots/baseline_singleDay.png}
\caption{Comparison of \trip variants with baseline}
\label{fig:baseline-single}
\end{figure}

\subsection{Comparison with Baseline}

We first experiment to see the performance of \trip against the baseline.
Figure~\ref{fig:baseline-single}\footnote{This and all subsequent graphs use a combination of line colors for utility variants and point shapes for transportation modes. Since there are 3 utility variants and 3 transportation modes, a combination of these legends produces the 9 \trip variants. For example, since Binary is represented by cyan-colored lines and Walking by circles, a cyan-colored line with circles represent the Binary Walking variant of \trip.}
shows the utility scores across different time budgets under a high cost budget scenario (100000 units) for the city of Osaka for a single day.
As expected, utility increases with increasing time budget, since there is more time to visit more POIs.
Notably, the \trip Binary-Walking variant closely replicates the behavior of the baseline model, validating its correctness and fairness for comparative purposes. The other models, especially the Hybrid mode ones, outperform the baseline significantly.
Since the cost budget is very high, the utility scores of Taxi match those of Hybrid.

%Overall, the plot effectively highlights how our approach—especially the allowance for partial POI visits and the use of flexible transport—leads to significantly improved itinerary recommendations, both in terms of utility and adaptability to real-world constraints.

\begin{figure}[t]
\includegraphics[width=\columnwidth]{plots/baseline_multiDay.png}
\caption{Comparison against baseline for multi-day trips}
\label{fig:baseline-multi}
\end{figure}

We next experiment with multi-day itinerary on Osaka for a very high cost budget of 100000 to ensure that cost is not a limiting factor for the baseline.
Each day was assigned a time budget of 8 hours.
Figure~\ref{fig:baseline-multi} shows the results for 1 to 4 days.
The baseline utilities were calculated by creating daily itineraries one after another, making sure not to repeat any POIs from previous days, and summing up the utilities.
The \trip variants run optimized multi-day planning models and, consequently, perform much better. Given the consistent and significant advantage of \trip over the baseline, we chose to leave the baseline out of future comparisons.

\begin{figure*}[t]
    \includegraphics[width=0.33\textwidth]{plots/exp1-osaka.png}
    \includegraphics[width=0.33\textwidth]{plots/exp1-budapest.png}
    \includegraphics[width=0.33\textwidth]{plots/exp1-delhi.png}
    \includegraphics[width=0.33\textwidth]{plots/exp1-glasgow.png}
    \includegraphics[width=0.33\textwidth]{plots/exp1-vienna.png}
    \includegraphics[width=0.33\textwidth]{plots/exp1-edinburgh.png}
    \caption{Utility variation with different time and cost budgets for 6 cities}
    \label{fig:cities}
\end{figure*}

\subsection{Multiple Cities}

Figure~\ref{fig:cities} shows the results of single day tour planning for 6 different cities under various time and cost budgets for different utility variants for the Hybrid mode.
The time budget is varied within a practical range of 480 to 600 minutes, and the cost budget is chosen to be appropriately aligned with the economic characteristics of the city.
As expected, higher cost and time budgets allow for higher utilities.
The Continuous utility variant shows the best performance.
\ab{can we have utilization numbers here?}
Since the trends remain the same across the cities, we choose Osaka as a representative, and show all subsequent results for it only.

%This offers a broad perspective on how the system adapts to varying resource constraints in diverse urban environments. For the remainder of the analysis, we focus on the city of Osaka as a representative case with a time budget of 480 minutes and cost budget of 6000 units. Unless stated otherwise, the default time budget is varied within a practical range of 480 to 600 minutes, and the cost budget is chosen to be appropriately aligned with the economic characteristics of the city. This approach allows us to maintain clarity while drawing generalizable conclusions from representative trends observed in one city.

\subsection{Effect of Multi-modality}

\begin{figure}[t]
\centering
\includegraphics[width=\columnwidth]{plots/multimodality1.png}
(a) Fixed cost budget of 6000 units 
%\label{fig:mm1}
\includegraphics[width=\columnwidth]{plots/multimodality2.png}
(b) Fixed time budget of 480 minutes
\caption{Effect of transportation modes}
\label{fig:multi-modal}
\end{figure}

We next evaluate the effect of multiple modes of transport (Figure~\ref{fig:multi-modal}).
For a fixed cost budget (of 6000 units), the top graph shows that Hybrid is generally the best mode by a significant margin. This is intuitive since Hybrid chooses the best of both the worlds.
The utility obtained using only Taxi saturates after a while, since once the fixed cost budget is exhausted, it cannot visit any more POI.
The bottom graph shows the effect of increasing cost budget for a fixed time budget of 480 minutes.
Again, while Hybrid is the best mode, with larger cost budgets, the Taxi variant catches up quickly.
This is intuitive since a large cost budget allows the traveler to quickly visit multiple POIs by spending more on taxi.
Since walking does not incur any cost, for a fixed time budget, there is no effect of the cost budget on it, and the utility remains the same.
Among the utility variants, Continuous shows the best performance.

\ignore{

Figure~\ref{fig:multi-modal} distinguishes utility computation variants using different line colors. The travel modes are represented by markers: circle for Walking, star for Taxi, and triangle for Hybrid mode. For instance, a blue star denotes $TRIP\_T\_B$ mode, orange triangle denotes $TRIP\_H\_S$ mode and so on.
\ab{what is new in this graph in addition to Fig 5?}

This experiment was done to illustrate the impact of varying time and cost budgets on the utility achieved by different transportation modes in our itinerary planning framework. The top graph explores how increasing the time budget (with a fixed cost budget of 6000 units) affects utility. Here, we observe that the utility continues to increase for walking and hybrid modes, while it saturates for the taxi-only variant. This is because the taxi mode relies heavily on cost availability—once the fixed cost budget is consumed, additional time offers very little, to no further improvement. In contrast, the hybrid mode, which leverages both walking and taxi flexibly, consistently outperforms the single-mode options across all three TRIP variants, highlighting the effectiveness of our multi-modality feature.

The lower graph, which holds the time budget constant at 480 minutes while increasing the cost budget, further supports this observation: hybrid models adapt more efficiently to increased resource availability, maximizing utility better than single-mode approaches. In this case, the utility for walking-only variants remains saturated across all TRIP variants, as the potential gains from walking are constrained by the fixed time rather than by cost. Furthermore, a consistent trend is observed wherein the CLF (Continuous Linear Function) scoring model outperforms the slab-based model. Together, both graphs validate our design choices, demonstrating the combined advantages of multi-modal transportation and continuous utility modeling in enhancing itinerary quality.\\

}

%\noindent\textbf{Time Utilization}

\subsection{Time and Cost Profile}

\begin{figure}[t]
\centering
\includegraphics[width=\columnwidth]{plots/tu1.png}
(a) Fixed time budget of 480 minutes (Continuous)
%\caption{Travel Times vs Visit Times in 3 travel Modes on different cost budgets and fixed time budget (480 minutes)}
%\label{fig:TimeUtilization}
%\end{figure}
%\begin{figure}[th]
% \includegraphics[width=0.23\textwidth]{plots/TIME_UTILIZATION_BINARY1.png}
\includegraphics[width=\columnwidth]{plots/tu2.png}
(b) Fixed cost budget of 6000 units
% \includegraphics[width=0.23\textwidth]{plots/TIME_UTILIZATION_SLABS1.png}
% \centering
% \includegraphics[width=0.25\textwidth]{plots/tu_legend2.png}
\caption{Time utilization (numbers on bars show utility scores)}
\label{fig:time-utilization}
\end{figure}

To understand the effects of different modes of transport even better, we next plot the time and cost utilization profiles.

Figure~\ref{fig:time-utilization} shows how the time is spent across visiting POIs versus traveling between POIs for different time and cost budgets.
Walking shows a high travel-to-visit time ratio, which is expected due to the slow nature of walking. A significant amount of time is, thus, spent in traveling. On the other extreme, Taxi minimizes travel time. Hybrid offers a solution which is mid-way, and achieves a more moderate ratio and a better utility.
Across all variants, increasing time budget allowed better utilization of available time, with Hybrid mode consistently providing the most balanced performance. This highlights that optimizing for utility involves not just maximizing visit duration but also strategically balancing travel efficiency with multi-modal flexibility.

\ignore{

Figures~\ref{fig:TimeUtilization} and ~\ref{fig:TimeUtilization1} present a comparison of travel time versus visit time across the three TRIP variants---walking-only (TRIP\_W), taxi-only (TRIP\_T), and hybrid (TRIP\_H)---at a fixed time budget of 480 minutes and varying cost budgets and at a fixed cost budget of 6000 units and varying time budgets respectively. Across all three utility scoring versions, a clear trend emerges: TRIP\_W exhibits a high travel-to-visit time ratio, reflecting the slower nature of walking as a mode of transport. In contrast, TRIP\_T minimizes travel time due to exclusive reliance on taxis, thereby maximizing time spent at points of interest (POIs). The hybrid model, TRIP\_H, maintains a balanced ratio between travel and visit time, offering a compromise between the two extremes. While the taxi-only model may seem efficient in terms of maximizing visit time, earlier experiments have demonstrated that it is the hybrid TRIP\_H that achieves the highest overall utility. Across all variants, increasing time budget allowed better utilization of available time, with Hybrid mode consistently providing the most balanced performance across travel and visit times. This highlights that optimizing for utility involves not just maximizing visit duration but also strategically balancing travel efficiency with multi-modal flexibility. 

}

Further, the Hybrid mode consistently achieves high time utilization, utilizing more than 95\% of the time budget, across all time and cost budgets. 
The Taxi mode shows better time utilization with smaller time budgets. For a higher time budget, since the cost budget is fixed, it quickly exhausts the cost and, consequently, fails to utilize the rest of the available time.
With increasing cost budget, however, it again improves the time utilization.

\ignore{

Walking mode shows similar time utilization across all cost budgets, indicating its inability to enhance utility on increasing cost budgets whereas on increasing time budget, its time utilization gets improved. Taxi mode starts strong with \textbf{98.7\%} utilization at 480 minutes, but gradually declines to around \textbf{74--82\%} at higher time budgets, as surplus time isn’t fully utilized because taxi is a cost based mode of transportation and hence on constant cost budget, increasing time budgets result in decline of time utilisation. This trend is evident in the plot, where hybrid mode bars nearly touch the top, reflecting optimal usage, while walking and taxi modes leave more unused time. Thus, hybrid mode offers the most consistent time utilization.
\\

}

%\noindent\textbf{Cost Utilization}

\begin{figure}[t]
\centering
% \includegraphics[width=\columnwidth]{plots/CU3_pkj.png}
\includegraphics[width=\columnwidth]{plots/cu5.png}
(a) Fixed time budget of 480 minutes
\includegraphics[width=\columnwidth]{plots/CU1.png}
(b) Fixed cost budget of 6000 units (Continuous)
% \includegraphics[width=\columnwidth]{plots/CU2_pkj.png}
%\caption{Travel Cost vs Visit Cost  in 3 travel Modes on different time budgets and fixed cost budget (6000 units)}
%\label{fig:CostUtilization1}
%\end{figure}
%\begin{figure}[th]
% \includegraphics[width=\columnwidth]{plots/CU4_pkj.png}
% \includegraphics[width=\columnwidth]{plots/CU6_pkj.png}
% \centering
% \includegraphics[width=0.3\textwidth]{plots/cu5_legend.png}
\caption{Cost utilization (numbers on bars show utility scores)}
\label{fig:cost-utilization}
\end{figure}

Figure~\ref{fig:cost-utilization} shows the corresponding cost utilization profiles for the same experiments.
The POI visit costs are typically lower than the travel costs.
Hybrid offers a better ratio since it can visit more POIs.
Cost utilization, however, is not always an indicator of utility maximization.
The cost of visiting a POI is not always proportional to its utility.
In Taxi mode, although faster travel allows reaching distant POIs, the high travel cost often exhausts the budget before accessing faraway high-utility POIs.

\ignore{

\noindent These plots in Figures~\ref{fig:CostUtilization1} and ~\ref{fig:CostUtilization2} highlight that in itinerary optimization, cost utilization and utility maximization do not always go hand in hand — the system’s goal is to maximize utility under given constraints rather than simply minimizing or maximizing cost. The entrance (visiting) cost is not directly proportional to the utility score, as some high-utility POIs may have lower entrance costs while others with higher cost POIs may not contribute proportionally to utility. In pure taxi mode, although faster travel allows reaching distant POIs, the high travel cost often exhausts the budget before accessing faraway high-utility POIs, leaving unutilized time. In contrast, the hybrid mode (walking + taxi) leverages walking to access distant, high-utility POIs without incurring excessive travel costs, leading to higher utility even when total cost utilization appears lower.

The Cost utilization varies across travel modes and budget settings. With a fixed cost budget of 6000 units, walking mode exhibits cost utilization around \textbf{30\%} whereas Taxi and Hybrid  modes leads to higher cost utilizations of greater than \textbf{{95\%}}, showing improved resource usage due to multimodal flexibility. With a fixed time budget of 480 minutes, cost utilization in Hybrid mode increases with budget. Evidently, the cost utilisation of walking mode saturates across all cost budgets because no cost is incurred in this travel mode for transportation and all cost is utilized in entrance fee which is relatively less, while taxi mode shows similar trend as hybrid mode because it is a costly mode of transport and it utilizes the cost budget effectively.\\

}

%\noindent\textbf{Utility vs Time budget on different cost budgets}\\

\ignore{

\noindent \textbf{How to Read the Graphs}

\noindent In figure~\ref{fig:cities}, the graph uses \textbf{colored lines} to represent different \textbf{cost budgets}—for example in Osaka we use blue for 5000 units, green for 10000 units, and red for 15000 units—while \textbf{symbols on the lines} denote different versions of the itinerary planner: circle for \textit{Binary}, triangle for \textit{Slab}, and star for \textit{Continuous Function (CF)}. Each line formed by a specific combination of color and symbol indicates the \textbf{utility trend} for a particular planner variant at a given cost budget. For instance, in Osaka the \textbf{blue line with square markers} represents the performance of the \textit{Binary version} at a \textbf{cost budget of 5000 units}. This combination-based encoding enables a comparative analysis of planner performance across various time and cost constraints.\\

}

\ignore{

\noindent\textbf{Insights}

\noindent Figure~\ref{fig:cities} illustrates how the achieved utility varies across different combinations of time and cost budgets for the six cities in our dataset. A few consistent trends emerge across all cities. First, for a fixed cost budget, increasing the available time consistently leads to higher utility, highlighting the value of longer exploration durations. Second, for a given time budget, allocating a higher cost budget also results in improved utility, indicating the benefit of greater financial flexibility.\\

}

% % \newpage
% \noindent\textbf{Variation of Utility in Multi-day Itineraries}
% \begin{figure}[th]
% \includegraphics[width=0.45\textwidth]{plots/multiday1_pkj.png}
% % \end{figure}
% % \begin{figure}[th]
% \includegraphics[width=0.45\textwidth]{plots/multiday2_pkj.png}
% \caption{Plots showing trends in Multi-day itineraries (3-day Trip) with Time Budget 480 minutes and Cost Budget 6000 units respectively}
% \label{fig:util_md}
% \end{figure}

% Figure ~\ref{fig:util_md} presents the utility trends for the multi-day variant of our itinerary planner for a 3 day trip. Similar to the single-day analysis, a consistent increasing trend in utility results is observed with fixed cost budget and increasing time budget and vice-versa, across the various formulations. Specifically, the utility achieved using the fractional variant is consistently higher than that of the binary variant, reaffirming the advantages of allowing partial visitations in enhancing overall experience. Furthermore, within the fractional formulations, the CLF (Continuous Linear Fractional) approach typically yields higher utility than the slab-based version, demonstrating its stronger capability to balance constraints and exploit budget flexibility effectively across multiple days of planning. However slabbed version works better than the binary version and thus it can be used to mathematically model the non linear utility functions in our problem which otherwise cannot be handled directly by ILP solvers and thus resulting in a better performance than the binary version.\\

\subsection{Multi-day Trips}

\begin{figure}[t]
\centering
\includegraphics[width=0.45\textwidth]{plots/multivssingle.png}
\caption{Benefits of solving a single multi-day optimization problem against aggregating over several single-day itineraries}
\label{fig:multi-day}
\end{figure}

We next measure the effect of optimizing the utility for a multi-day itinerary against solving for single days, and summing the utilities.
Figure~\ref{fig:multi-day} shows the utility scores for the various settings for two utility variants (the Slab variant shows similar results as Continuous and is, hence, omitted).
The first bar shows the different utilities for each of the 3 days. The POIs visited in earlier days are set as must-avoid later.
The second bar solves a one-shot 2-day itinerary problem, while the third bar solves the same for a 3-day itinerary.
The time budget is set to 480 minutes per day.
The cost budget for a single day itinerary is 2000 units; thus, for a 2-day (respectively, 3-day) multi-day itinerary, it is set to 4000 (respectively, 6000 units).
As can be clearly shown, solving a single optimization problem for multiple days shows a much larger utility (10-15\% more) over solving for 3 separate days and aggregating the utilities.
Even for 2 days, the gain in utility is around 15\%.

\ignore{

Figure~\ref{fig:multi-day} illustrates a comparison of total utility achieved with different trip planning configurations: aggregated 3 single-day trips versus actual 2-day and 3-day multi-day itineraries, optimized for both Binary and CLF versions. In this experiment, each day was assigned an equal time budget of 480 minutes. For single-day trips, the cost budget was uniformly set at 2000 units per day, and for the multi-day cases, the aggregate cost budgets were similarly scaled: 4000 units for 2-day and 6000 units for 3-day trips. As a basis of comparison, the 3 single day trips were: source-to-hotel travel (bottom segment), hotel-to-destination travel (middle segment), and hotel-to-hotel travel (top segment). The last two segments are swapped for a fair comparison with 2-Day multiday trip version.\\

The results clearly demonstrate that multi-day planning consistently outperforms naive aggregation over single days. In the Binary variant, the 2-day multi-day plan has a score of 4084, whereas two single-day plans aggregated give lower cumulative utility. Similarly, in the 3-day variant, the multi-day plan has a score of 5349, much higher than the three single-day utilities summed up. The same is the case with the CLF variant, where the multi-day plans (4218 for 2-day, 5382 for 3-day) consistently beat the aggregated single-day utilities. This improved performance is due to the flexibility and optimization advantages of multi-day planning, where activities and itineraries can be optimized globally over multiple days as opposed to being restricted within independent daily plans. Thus, multi-day trip planning is not a single-day plan aggregation but a richer solution space that optimally exploits inter-day dependencies.\\

}

\subsection{Personalized Constraints}

\begin{figure}[t]
\centering
\includegraphics[width=0.24\textwidth]{plots/mustsee.png}
\includegraphics[width=0.24\textwidth]{plots/mustavoid.png}
\includegraphics[width=0.24\textwidth]{plots/ordering.png}
\includegraphics[width=0.24\textwidth]{plots/category.png}
% \raisebox{0.5\height}{\includegraphics[width=0.24\textwidth]{plots/legend_personalized_pkj.png}}
\caption{Various personalized constraints}
\label{fig:personalizedconstraints}
\end{figure}

Figure~\ref{fig:personalizedconstraints} shows the results of various personalized constraints over the 3 utility variants.
In this study, a cost budget of 6000 units and a time budget of 480 minutes is fixed across all the cases.
With multiple must-see POIs fixed by the traveler, the utility falls, since there is not much scope for the solver to optimize. The same happens with multiple must-avoid POIs, although the decline is less sharp.
The same reduction in utility is observed with more number of ordering and category constraints.
Binary shows the sharpest drops since it has the least leeway; Slab and Continuous variants can partially visit a POI, and move to other POIs for better utility gain.

\ignore{

The impact of incorporating personalized constraints, that is, must-see, ordering, and category and must-avoid constraints, on overall utility is illustrated in the four graphs of figure ~\ref{fig:personalizedconstraints}. In this study a cost budget of 6000 units and a time budget of 480 minutes is fixed across all cases, it is evident that increasing the number of constraints leads to a decline in the achievable utility, as the solution space becomes more restricted. However, the extent of this reduction varies across the different formulations. Specifically, the CLF and slab variants show a relatively gradual decline in utility compared to the binary variant. This is primarily because the binary model lacks the flexibility to accommodate partial visits to POIs, thereby limiting its ability to adapt to tighter constraints. In contrast, the CLF and slab variants can better navigate these constraints by leveraging their ability to assign fractional visits, thus preserving higher utility under increasingly personalized user preferences.\\

}

%\noindent\textbf{Effect of Dynamic Re-planning}

\subsection{Dynamic Re-planning}

\begin{figure}[t]
    \centering
    \includegraphics[width=0.50\textwidth]{plots/dynamic_pkj.png}
    \caption{Comparison of utility when tourist spends less or more time at a POI}
    \label{fig:dynamic}
\end{figure}

% \begin{figure*}[htbp]
%     \centering
%     % \includegraphics[width=\textwidth]{plots/scalability_pkj.png}
%     \includegraphics[width=\textwidth]{plots/scalability_new_pkj.png}
%     \caption{Time of computation on increasing number of POIs in single-day trips.}
%     \label{fig:scalability1}
% \end{figure*}

%\noindent\textbf{1. How to Read ~\ref{fig:dynamic_pkj}}

The next experiment captures the changes in utility when dynamic re-planning is done.
Figure~\ref{fig:dynamic} covers five different scenarios for a POI, according to the time spent in visiting it: (1)~\emph{50\% less}, i.e., when the traveler spends 50\% of the visit time prescribed, (2)~\emph{25\% less}, i.e., 75\% of the prescribed time, (3)~\emph{ontime}, i.e., when the traveler spends the exact prescribed time, (4)~\emph{50\% more}, i.e., 150\% of the prescribed time, and (5)~\emph{100\% more}, i.e., 200\% of the prescribed time.
The different groups specify the \emph{order} of the POI where the change is made (all other POIs spent the exact prescribed time).
The time and cost budget are 480 minutes and 6000 units respectively.
The optimal static itinerary consists of 4 POIs.
Our planner dynamically recalculates the itinerary by considering the actual time a user spends at each Point of Interest (POI) and the time taken to travel between them, in contrast to the initially suggested schedule.

The figure shows that the effect of spending a substantially different time on the first POI does not have much effect, since there is enough time for the planner to find another itinerary with a similar utility.
Similarly, spending less time in the third POI does not give a much better itinerary since there is not much scope to maneuver, given the time budget already spent. Spending a lot of extra time may, however, decrease the utility significantly, especially if this does not allow reaching the fourth and subsequent POIs.
The most amount of variability is visible for the second POI.
Spending lesser times results in re-planning significantly better itineraries, while spending much longer times results in lesser utility itineraries, despite re-planning.

Overall, this variation in utility values underscores the importance of adapting the itinerary based on real-time user behavior, demonstrating the effectiveness of our dynamically responsive planning approach.

\ignore{

%In all cases, travel times between any pair of POIs remain unaffected by user behaviour, and it takes the same time to travel by the user as suggested. Each POI has five bars, each representing different scenarios as stated above. Bar heights indicate utility levels, with values labeled on top. Distinct patterns and colors differentiate time-spent categories, as indicated in the legend.\\

%\noindent\textbf{2. Insights}

% \pri{Discuss about binary variant usage here}
Our planner dynamically recalculates the itinerary by considering the actual time a user spends at each Point of Interest (POI) and the time taken to travel between them, in contrast to the initially suggested schedule.  As shown in figure, at a time budget of 480 minutes and cost budget of 6000 units in TRIP\_H\_B variant, the \textbf{utility consistently decreases} as the time spent at a POI increases.

This trend can be explained as follows:

\begin{itemize}
    \item Spending \textbf{less time} at a POI allows the user to \textbf{save time}, which can be used to visit \textbf{additional POIs}, thereby increasing the overall utility.
    \item Conversely, as more time is spent at a single POI, the user has \textbf{less time available} to explore other POIs, resulting in a \textbf{decrease in total utility}.
\end{itemize}

\noindent This variation in utility values underscores the importance of adapting the itinerary based on real-time user behavior, demonstrating the effectiveness of our dynamically responsive planning approach.\\

}

%\noindent\textbf{Scalability}

\subsection{Scalability}

In this section, we test the scalability of our MILP solution in terms of running time for various parameters.

Figure~\ref{fig:number-of-pois} and Figure~\ref{fig:number-of-days} show that the time increases exponentially with number of POIs and number of days in the trip, which is expected for a linear programming solution.
Importantly, the computation times remain \emph{practical} even when the number of POIs or the number of days is very high.

\ignore{

To evaluate the scalability of our itinerary planning system, we examine how the computation time required to generate a complete itinerary varies with the number of Points of Interest (POIs) in a city, while keeping the time and cost budgets fixed at sufficiently accommodating values of 480 minutes and 6000 units respectively. As illustrated in Figure~\ref{fig:dynamic}, the observed increase in computation time with the number of POIs is expected, given that our approach is formulated as an Integer Linear Program (ILP). This growth aligns with the theoretical complexity of the underlying problem — a variant of the Team Orienteering Problem — which is known to be NP-Hard. Despite this, our formulation remains tractable for problem sizes typical of real-world tourist cities. Despite this theoretical complexity, our system demonstrates strong practical performance. For instance, with approximately 180 POIs, the itinerary is generated in under 1 minute. Even with nearly 260 POIs, the computation time remains approximately 5 minutes. These results highlight that, for realistically sized cities, our system maintains high computational efficiency, making it a viable and scalable solution for real-world deployment.
% \pri{was the content for optimal utility vs AI added?}

}

\begin{figure}[t]
    \centering
    \includegraphics[width=\columnwidth]{plots/scalability_new_pkj.png}
    \caption{Scalability with number of POIs}
    \label{fig:number-of-pois}
\end{figure}

\ignore{

In order to further check the scalability of the suggested itinerary planning system, we designed an experiment by increasing the number of days in the trip as shown in figure~\ref{fig:scalability2}. The time budget was chosen as 480 minutes per day and the cost budget for day 1 was kept 2000 units for the whole trip. We kept increasing the cost budget by 2000 units on each subsequent day. The experiment was run for trips from 1 to 6 days.

The Time of Computation (TOC) for each case was measured. As anticipated, the TOC grows large as the days increase, mostly owing to the exponential increase in the solution space with the more days. It thereby aptly captures the scalability issue while the computation time still remains scalable for a practical constraint.\\
\nl{Please read this once}

}

\begin{figure}[th]
    \centering
\includegraphics[width=\columnwidth]{plots/multidayvstoc.png}
     \caption{Scalability with number of days in multi-day trips}
    \label{fig:number-of-days}
\end{figure}

% \begin{figure*}[htbp]
%     \centering
%     \includegraphics[width=\textwidth]{plots/costbudgetvstoc.png}
%     \caption{Time of computation on increasing cost budget at fixed time budget 480 minutes}
%     \label{fig:costbudgetvstoc}
% \end{figure*}



% \begin{figure*}[htbp]
%     \centering
%     \includegraphics[width=\textwidth]{plots/timebudgetvstoc.png}
%     \caption{Time of computation on increasing time budget at fixed cost budget 6000 units}
%     \label{fig:timebudgetvstoc}
% \end{figure*}

We next show the change in computation time when the cost and time budget change.
As shown in Figure~\ref{fig:cost-budget}, for extremely low cost budgets, the time taken is small, but it increases rapidly with increasing cost budget. This is due to the fact that more choices become available when the cost budget increases. However, once the cost budget exceeds a threshold, the solver can effectively ignore about it, since almost all solutions become feasible in terms of cost.
Similarly, the computation time remains very low, unless the time budget is extremely high (Figure~\ref{fig:time-budget}) since only then almost all solutions become feasible in terms of time.

\ignore{

\noindent The computational efficiency of the suggested itinerary optimization model was tested against both cost budget and time budget changes. In figure~\ref{fig:costbudgetvstoc}, as the cost budget rises from low values, the computation time records a minimal increment and hits a plateau, perhaps as the solution space complexity is raised. But above some cost budget limit (approximately 6000), the computation time levels off and is always low for a broad range of cost budgets. This indicates that after the feasible space is large enough, further increases in the budget no longer have a major effect on solver performance. On the contrary, figure~\ref{fig:timebudgetvstoc} illustrates that computation time is much less robust to the time budget. While initially the increase in time budget leads to only a gradual rise in computation time, beyond 900 minutes, the solver time escalates sharply, indicating the rapidly growing complexity of exploring a significantly larger feasible space. Particularly towards the upper end, computation time skyrockets, pointing out that cost budget is less important than time budget in defining the computational complexity of the problem.

}

\begin{figure}[t]
    \centering
    \includegraphics[width=\columnwidth]{plots/costbudgetvstoc.png}
    \caption{Time of computation on increasing cost budget at fixed time budget 480 minutes}
    \label{fig:cost-budget}
\end{figure}

\begin{figure}[t]
    \centering
    \includegraphics[width=\columnwidth]{plots/timebudgetvstoc.png}
    \caption{Time of computation on increasing time budget at fixed cost budget 6000 units}
    \label{fig:time-budget}
\end{figure}

%\noindent\textbf{Conclusions}\\

%\ab{Please read this section!}\\

%As shown in Figure~5, we are able to successfully replicate the existing baseline version (\texttt{TRIP\_W\_B}).

\subsection{Summary of Experimental Results}

Overall, our experimental results provide the following insights.
The \trip variants outperform the baseline significantly.
The Hybrid is the best mode of transport.
The Continuous utility variant gives slightly better results than the Slab variant, and both of them outperform the Binary variant due to lack of flexibility for the Binary variant.
Optimizing for multiple days in one shot gives significantly better results than solving for single days, and adding them up.
Personalized constraints are handled efficiently by our \trip solution.
Dynamic re-planning can increase the utility when there is scope to maneuver the rest of the plan.
Finally, \trip solver, despite being a MILP solution, is practical.

\ignore{

Moving forward, as observed in all the figures of this section, both the Continuous and Slab variants of the itinerary planner demonstrate superior performance over the Binary variant by enabling users to achieve higher utility values. This provides sufficient evidence to establish that the Continuous and Slab variants are more effective than the Binary variant. However, while the Continuous variant generally yields higher utility compared to the Slab variant, it is not appropriate to directly compare these two approaches. The Slab variant was specifically introduced to address the challenge of modeling non-linear utility functions, which cannot be directly handled by ILP solvers. By discretizing such non-linear functions into multiple slabs, they can be mathematically formulated and optimized within the ILP framework. Therefore, Continuous and Slab variants serve distinct purposes and address different problem complexities; any direct comparison in terms of utility alone would be misleading. Both variants offer unique advantages in handling different utility modeling scenarios, and hence, we continue the subsequent analysis with both, without declaring either one superior over the other.

}

\section{Conclusions and Future Work}

This paper revisits the trip itinerary planning problem and proposes a novel solution framework called \trip that returns an optimal itinerary as a solution. This solution allows multi-day itinerary planning that not only considers the tourist's starting and ending locations and timings, but also adheres to the operational schedule of each POI. The proposed solution is unique and powerful due to its ability to accommodate multiple transportation modes, factoring of user specified personalized constraints such as must-see constraints, ordering constraints and category constraints, consideration of multiple utility variants, and the capability of dynamic re-routing of the itinerary to account for unplanned delays or early exits experienced during the previously visited POIs. Empirical evaluation on several popular destinations show the   efficacy and efficiency of the proposed solution. In the future, other transportation modes such as cycle, public bus and metro can be considered. Further, the real-time crowd and the resulting waiting time at each POI can be factored to generate more useful itineraries.


\pagebreak

\section*{AI-Generated Content Acknowledgement}

We have used the generative artificial intelligence (GenAI) tool ChatGPT for help in paraphrasing and small edits.

%\bibliographystyle{ACM-Reference-Format}
\bibliographystyle{IEEEtranS}
\bibliography{references}

\end{document}

