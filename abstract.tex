\begin{abstract}
	%
	Given a tourist who intends to visit a set of points of interest (POIs)
	spread across a geographical region, the trip itinerary planning
	 problem aims to identify an \textit{optimal itinerary} that \textit{maximizes} the
	tourist's utility under a specified  cost and time budget. An itinerary is
	defined as an ordered sequence of POIs that adheres to the time and budget
	constraints. This problem is not only important for tourists, but also for
	tour planners that offer personalized trips as business. Although there are
	few prior works that have attempted to address the above problem, they allow
	limited flexibility in terms of accommodating multiple transportation modes, planning a multi-day itinerary that adheres to the operational schedule of the POIs, 
	re-planning the itinerary based on tourist's actual visit duration of the
	POIs and the live traffic situation, customizing the itinerary based on the
	personalized constraints and the utility function chosen by the user. This
	work revisits this problem with the goal of overcoming the above
	limitations and considering more realistic factors.
	%
	In particular, the proposed solution, named \emph{TRIP (TRip Itinerary Planner)} allows the tourist to (1) plan a \textit{multi-day
	itinerary} that considers the operational schedule of the POIs and user's choice of the start and the end time of the trip and the originating POI and the ending POI for each day;
 (2) factor in \textit{multiple transportation modes} based on available
	cost and time budget; (3) customize the itinerary by allowing the tourist to specify one or more  \textbf{personalized constraints} that include must-see constraints, ordering constraints and category constraints.
(4) choose from a
	\textit{set of utility function variants} that best captures his/her travel
	experience;
(5) \textbf{dynamically} adjust the remaining itinerary based on the delays or early exits made during the previously visited POIs.
  The problem is modeled using a directed graph, where the nodes represent the
	POIs and the edges represent the available transportation modes between each
	pair of POIs. Subsequently, this problem is solved using a mixed integer
	linear program (MILP) that returns the optimal itinerary. A comprehensive
	empirical evaluation over a real data set comprising of several popular
	tourist destinations demonstrate the efficacy and efficiency of the proposed
	solution.
	%
\end{abstract}

