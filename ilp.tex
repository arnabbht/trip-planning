\section{The \trip Solution}
\label{sec:ilp}

In this section, we first outline the basic solution, before expanding it for
the different extensions.
\ab{write about optimality}

\subsection{Basic Solution}
\label{sec:basic}

We propose a \emph{mixed integer linear programming (ILP)} solution for
the \trip problem.  Each POI $v_{j}$ is associated with a binary
variable $y_{j}$, which is $1$ if it is included in the optimal
itinerary, and $0$ otherwise.  Each edge (i.e., a connection between
POIs and the source and destination) between two POIs $v_{j}$ and
$v_{k}$ is also associated with an edge binary variable $z_{j,k}$; it is
$1$ if the edge is chosen, and $0$ otherwise.  To denote $m$ different
transport modes, $z$ is superscripted with the mode.  Thus, walking,
taxi, etc. are denoted as $z^1$, $z^2$, etc.  Assuming the city has $N$
POIs, without loss of generality, we treat the source $S$ and
destination $D$ as POI $v_{0}$ and $v_{{N+1}}$ respectively for uniform
treatment.  The cost and time of visit of a POI $v_i$ is denoted by
$c_i$ and $t_i$ respectively, while the cost and time of traveling from
POI $j$ to POI $k$ is denoted by $c_{j,k}$ and $t_{j,k}$ respectively.

\begin{align}
	\label{eq:max}
	\max \quad & \sum_{\forall i} U_{{i}} \cdot y_{i} \\
	\text{subject to } \quad & \nonumber \\
	\label{eq:y}
	y_{i} & \in \{ 0, 1 \}, \forall i \\
	\label{eq:z}
	z^m_{j,k} & \in \{ 0, 1 \}, \forall j,k, \forall m \\
	\label{eq:sd}
	y_{0} = y_{{N+1}} & = 1 \\
	\label{eq:cost}
	\sum_{\forall j, k, \forall m} c_{{j},{k}} \cdot z^m_{j,k} + \sum_{\forall i} c_{i} \cdot y_{i} & \leq C \\
	\label{eq:time}
	\sum_{\forall j, k, \forall m} t_{{j},{k}} \cdot z^m_{j,k} + \sum_{\forall i} t_{i} \cdot y_{i} & \leq T \\
	\label{eq:sout}
	\sum_{\forall k} z^m_{{0},{k}} & = 1, \forall m \\
	\label{eq:sin}
	\sum_{\forall j} z^m_{{j},{0}} & = 0, \forall m \\
	\label{eq:din}
	\sum_{\forall j} z^m_{{j},{N+1}} & = 1, \forall m \\
	\label{eq:dout}
	\sum_{\forall k} z^m_{{N+1},{k}} & = 0, \forall m \\
	\label{eq:out}
	\sum_{\forall j} z^m_{{j},{k}} & \leq 1, \forall k, \forall m \\
	\label{eq:in}
	\sum_{\forall k} z^m_{{j},{k}} & \leq 1, \forall j, \forall m \\
	\label{eq:mode}
	\sum_{\forall m} z^m_{{j},{k}} & \leq 1, \forall j, k \\
	\label{eq:connectivity}
	\sum_{\forall j, \forall m} z^m_{{j},{i}} = \sum_{\forall k, \forall m} z^m_{{i},{k}} & = y_i, \forall i \\
	\label{eq:nodechosen1}
	\sum_{\forall m} z^m_{{j},{k}} & \leq y_j, \forall j,k \\
	\label{eq:nodechosen2}
	\sum_{\forall m} z^m_{{j},{k}} & \leq y_k, \forall j,k \\
	\label{eq:edgechosen1}
	\sum_{\forall k, \forall m} z^m_{{j},{k}} & \geq y_j, \forall j \\
	\label{eq:edgechosen2}
    \sum_{\forall j, \forall m} z^m_{{j},{k}} & \geq y_k, \forall k \\
	\label{eq:self}
	z^m_{{i},{i}} & = 0, \forall i, \forall m  \\
	\label{eq:subtour}
	(s_j + t_{j} + t_{j,k}) \cdot z_{j,k} & \leq s_k, \forall j,k
\end{align}

%\ab{does this capture subset deletion?}
%\ab{how is ordering handled?}
%\ab{are there other constraints in basic problem?}

The above ILP solves the basic \trip problem.  Eq.~\eqref{eq:max}
represents the \emph{utility maximization} problem, where the
utility of a POI $i$ is denoted by $U_i$.  Eq.~\eqref{eq:y} marks
the choice of POIs chosen for the optimal itinerary; however,
Eq.~\eqref{eq:sd} mandates the choice of the source and destination.
Eq.~\eqref{eq:z} marks the choice of edges for the optimal
itinerary.  Eq.~\eqref{eq:cost} and Eq.~\eqref{eq:time} put the
cost and time \emph{budget constraints} by adding up the
respective values for the POIs that are chosen (corresponding to
$y = 1$).  Eq.~\eqref{eq:sout} and Eq.~\eqref{eq:din} ensure that
the source and the destination is connected to one and only one
POI in the \emph{correct direction}, i.e., via one outgoing edge
and one incoming edge respectively only.  Eq.~\eqref{eq:sin} and
Eq.~\eqref{eq:dout} ensure that there is no incoming edge to the
source and no outgoing edge from the destination.
Eq.~\eqref{eq:out} and Eq.~\eqref{eq:in} constrain that each POI
has at most one incoming and at most one outgoing edge.
Eq.~\eqref{eq:mode} ensures that the chosen mode of travel is at
most one of the $m$ possibilities.  Eq.~\eqref{eq:connectivity}
imposes the \emph{connectivity constraint} by ensuring that a
node, if chosen, has exactly one incoming edge and an outgoing
edge.  Eq.~\eqref{eq:nodechosen1} and Eq.~\eqref{eq:nodechosen2}
imposes the condition that if a node is chosen, corresponding
edges in and out of it must be present as well.
Eq.~\eqref{eq:edgechosen1} and Eq.~\eqref{eq:edgechosen2} ensures
the opposite, i.e., if an edge is chosen, the corresponding nodes
are chosen as well.  Finally, Eq.~\eqref{eq:self} ensures that
there is \emph{no self-loop}.
\ab{no subtour}


\ignore{

\begin{align}
    & \textbf{maximize} \sum_{d \in \text{days}} \sum_{i = 1}^{N} U(v_i) \cdot y_{i,d} \hspace{3.4cm} \ref{eq:multi_day_binary} \notag\\
    & \textbf{subject to:} \notag \\ 
    & \sum_{d \in \text{days}} y_{i,d} \leq 1 \quad \forall i \in \text{poi\_ids} \setminus \{\mathit{hotel\_id}\}\\
    & \text{day\_visit}_i = \sum_{d \in \text{days}} d \cdot y_{i,d}\quad \forall i \in \text{poi\_ids} \\
    & z_{i,j,d} = w_{i,j,d} + x_{i,j,d} \quad\forall i, j \in \text{poi\_ids},\, i \ne j,\; \forall d \in \text{days} \\
    & z_{i,j,d} \leq 1 \quad \forall i, j \in \text{poi\_ids},\, i \ne j,\; \forall d \in \text{days} \\
    &  z_{i,j,d} \leq y_{i,d} \quad \forall d \in \text{days},\quad \forall i, j \in \text{poi\_ids},\; i \ne j\  \\
    &  z_{i,j,d} \leq y_{j,d}, \quad \forall d \in \text{days},\quad \forall i, j \in \text{poi\_ids},\; i \ne j\ \\   
    & s_{i,d} \leq day\_start\_time + day\_budget + (1 - y_{i,d}) \cdot M \notag\\
    &\forall i \in \text{poi\_ids}, \forall d \in \text{days}\  \\
    & \textbf{1. First Day } (d = d_1): \notag \\
    & \sum_{\substack{j \in \text{POIs} \\ j \neq \text{starting-poi}}} z_{\text{source-poi},j,d_1} = 1 \\
    & \sum_{\substack{i \in \text{POIs} \\ i \neq \text{hotel}}} z_{i,\text{hotel},d_1} = 1  \\
    & \sum_{\substack{j \in \text{POIs} \\ j \neq \text{hotel}}} z_{\text{hotel},j,d_1} = 0  \\
    & \textbf{2. Last Day } (d = d_T): \notag \\
    & \sum_{\substack{j \in \text{POIs} \\ j \neq \text{hotel}}} z_{\text{hotel},j,d_T} = 1 \\
    & \sum_{\substack{i \in \text{POIs} \\ i \neq \text{ending-poi}}} z_{i,\text{ending-poi},d_T} = 1 \\
    & \sum_{\substack{j \in \text{POIs} \\ j \neq \text{ending-poi}}} z_{\text{ending-poi},j,d_T} = 0 \\
    & \textbf{3. Intermediate Days } (d \in \text{days} \setminus \{d_1, d_T\}): \notag \\[0.2cm]
    & \sum_{\substack{j \in \text{POIs} \\ j \neq \text{hotel}}} z_{\text{hotel},j,d} = 1 \quad \forall d \in \text{days} \setminus \{d_1, d_T\} \\
    & \sum_{\substack{i \in \text{POIs} \\ i \neq \text{hotel}}} z_{i,\text{hotel},d} = 1 \quad \forall d \in \text{days} \setminus \{d_1, d_T\} \\
    & \sum_{\substack{i \in \text{POIs} \\ i \neq k}} z_{i,k,d} = y_{k,d}
    \quad \forall d \in \text{days},\forall k \in \text{POIs} \setminus \{\text{start}, \text{end}, \text{hotel}\}\\
    & \sum_{\substack{j \in \text{POIs} \\ j \neq k}} z_{k,j,d} = y_{k,d}, \quad \forall d \in \text{days}, \forall k \in \text{POIs} \setminus \{\text{start}, \text{end}, \text{hotel}\}
\end{align}
\begin{align}
    & \sum_{i \ne j} t^{w}_{i,j} \cdot w_{i,j,d} + \sum_{i \ne j} t^{t}_{i,j} \cdot x_{i,j,d} + \sum_{i \in V} \text{t}(v_i) \cdot y_{i,d} \nonumber \leq H \hspace{0.9cm} \ref{mul_day_9}\\
    & \sum_{d \in \text{days}} \sum_{(i,j) \in I} c^{w}_{i,j} \cdot w_{i,j} + c^{t}_{i,j} \cdot x_{i,j} + \sum_{d \in \text{days}} \sum_{i \in I} c(v_i) \leq B \hspace{0.3cm} \ref{mul_day_25} \notag\\
    & s_{i,d} \geq \left( s_{j,d} + t(v_j) + t^{w}_{j,i,d} \cdot w_{j,i,d} + t^{x}_{j,i,d} \cdot x_{j,i,d} \right) \cdot z_{j,i,d} \notag\\
    &- M_{ji} \cdot (1 - z_{j,i,d}) \quad\forall i \neq j,\; \forall d \in \text{days} \\
    & ot(v_i) \leq s_{i,d} \leq ct(v_i) - t(v_i) \quad \forall v_i \in V, \forall d \in \text{days} \hspace{1cm}\ref{mul_day_30} \notag\\
    & y_{i,d} = 0 \quad \forall i \in \text{POIs}, \forall d \in \text{days} \notag \\& \text{ if } \texttt{day\_availability}[trip\_weekdays][d]][i] = 0 \hspace{1cm}\ref{mul_day_31} \notag\\
    & s_{start\_node,d} = \text{day\_start\_time} \quad \forall d \in \text{days}
\end{align}

The objective function (1) maximises the overall utility of the trip. Constraint (7) ensures that each POI is visited atmost once in the whole trip. Constraint (8) assigns a day to each POI on which it is visited during the itinerary. Constraints (9) and (10) ensure that only one mode out of Taxi and Walk is selected. Constraints (11) and (12) establish a logical connection between edges and vertices by preventing isolated edges and vertices. Constraint (13) puts an upper bound on the arrival time of each POI. Constraints (14),(15) and (16) make sure that the itinerary of first day is correctly modelled from source POI to hotel POI. Similarly, Constraints (17),(18) and (19) make sure that the itinerary of last day is correctly modelled from hotel POI to destination POI, and Constraints (20) and (21) help in modelling the itinerary of intermediate days from hotel to hotel correctly. Constraints (22) and (23) are connectivity constraints, which impose a restriction that if an edge enters a vertex, another edge must leave that vertex. Constraints (2),(3),(5),(6) are time budget, cost budget, opening closing time and opening closing day constraints respectively that are already explained in previous section. Constraint (24) ensures that if an edge goes from POI j to POI i, then the arrival time of i is greater than or equal to the sum of visit time of POI j and travel time between j to i. M is a large constant which relaxes the constraint when both of them aren't part of itinerary. Constraint (25) ensures that each day's itinerary starts on correct time as stated by the tourist.

}

