\section{The \trip Solution}
\label{sec:ilp}

In this section, we first outline the basic solution, before expanding it for
the different extensions.
Our solution is based on \emph{mixed integer linear programming} and is, hence,
\emph{optimal}.
In other words, no other algorithm can get a strictly better solution in terms
of utility with the given constraints.

\subsection{Basic Solution}
\label{sec:basic}

We propose a \emph{mixed integer linear programming (MILP)} solution for
the \trip problem.  Each POI $v_{j}$ is associated with a binary
variable $y_{j}$, which is $1$ if it is included in the optimal
itinerary, and $0$ otherwise.  Each edge (i.e., a connection between
POIs and the source and destination) between two POIs $v_{j}$ and
$v_{k}$ is also associated with an edge binary variable $z_{j,k}$; it is
$1$ if the edge is chosen, and $0$ otherwise.  To denote $m$ different
transport modes, $z$ is superscripted with the mode.  Thus, walking,
taxi, etc. are denoted as $z^1$, $z^2$, etc.  Assuming the city has $N$
POIs, without loss of generality, we treat the source $S$ and
destination $D$ as POI $v_{0}$ and $v_{{N+1}}$ respectively for uniform
treatment.  \ignore{ The cost and time of visit of a POI $v_i$ is denoted by
$c_i$ and $t_i$ respectively, while the cost and time of traveling from
POI $j$ to POI $k$ is denoted by $c_{j,k}$ and $t_{j,k}$ respectively.}

The cost and time of visit of a POI $v_i$ is denoted by
$c_i$ and $t_i$ respectively, while the cost and time of traveling from
POI $j$ to POI $k$ via mode $m$ is denoted by $c^{m}_{j,k}$ and $t^{m}_{j,k}$ respectively.

In addition, 
we introduce an \emph{arrival time} variable for each POI, denoted by $a_i$ for node $v_i$.

\begin{figure}[t]
\begin{align}
	\label{eq:max}
	\max \quad & \sum_{\forall i} U_{{i}} \cdot y_{i} \\
	\text{subject to } \quad & \nonumber \\
	\label{eq:y}
	y_{i} & \in \{ 0, 1 \}, \forall i \\
	\label{eq:z}
	z^m_{j,k} & \in \{ 0, 1 \}, \forall j,k, \forall m \\
	\label{eq:sd}
	y_{0} = y_{{N+1}} & = 1 \\
	\label{eq:cost}
	\sum_{\forall j, k, \forall m} c^{m}_{{j},{k}} \cdot z^m_{j,k} + \sum_{\forall i} c_{i} \cdot y_{i} & \leq C \\
	\label{eq:time}
	\sum_{\forall j, k, \forall m} t^{m}_{{j},{k}} \cdot z^m_{j,k} + \sum_{\forall i} t_{i} \cdot y_{i} & \leq T \\
	\label{eq:sout}
	\sum_{\forall k} z^m_{{0},{k}} & = 1, \forall m \\
	\label{eq:sin}
	\sum_{\forall j} z^m_{{j},{0}} & = 0, \forall m \\
	\label{eq:din}
	\sum_{\forall j} z^m_{{j},{N+1}} & = 1, \forall m \\
	\label{eq:dout}
	\sum_{\forall k} z^m_{{N+1},{k}} & = 0, \forall m \\
	\label{eq:out}
	\sum_{\forall j} z^m_{{j},{k}} & \leq 1, \forall k, \forall m \\
	\label{eq:in}
	\sum_{\forall k} z^m_{{j},{k}} & \leq 1, \forall j, \forall m \\
	\label{eq:mode}
	\sum_{\forall m} z^m_{{j},{k}} & \leq 1, \forall j, k \\
	\label{eq:connectivity}
	\sum_{\forall j, \forall m} z^m_{{j},{i}} = \sum_{\forall k, \forall m} z^m_{{i},{k}} & = y_i, \forall i \\
	\label{eq:nodechosen1}
	\sum_{\forall m} z^m_{{j},{k}} & \leq y_j, \forall j,k \\
	\label{eq:nodechosen2}
	\sum_{\forall m} z^m_{{j},{k}} & \leq y_k, \forall j,k \\
	\label{eq:edgechosen1}
	\sum_{\forall k, \forall m} z^m_{{j},{k}} & \geq y_j, \forall j \\
	\label{eq:edgechosen2}
    \sum_{\forall j, \forall m} z^m_{{j},{k}} & \geq y_k, \forall k \\
	\label{eq:self}
	z^m_{{i},{i}} & = 0, \forall i, \forall m  \\
	\label{eq:subtour}
	a_j + \sum_{\forall m}(t_{j} + t^{m}_{j,k}) \cdot z^{m}_{j,k} & \leq a_k, \forall j,k
\end{align}
	\figcaption{Basic mixed integer linear programming solution}
\label{fig:basic}
\end{figure}

Fig.~\ref{fig:basic} outlines the equations for the
MILP solution for the basic \trip problem.  Eq.~\eqref{eq:max}
represents the \emph{utility maximization} problem, where the
utility of a POI $i$ is denoted by $U_i$.  Eq.~\eqref{eq:y} marks
the choice of POIs chosen for the optimal itinerary; however,
Eq.~\eqref{eq:sd} mandates the choice of the source and destination.
Eq.~\eqref{eq:z} marks the choice of edges for the optimal
itinerary.  Eq.~\eqref{eq:cost} and Eq.~\eqref{eq:time} put the
cost and time \emph{budget constraints} by adding up the
respective values for the POIs that are chosen (corresponding to
$y = 1$).  Eq.~\eqref{eq:sout} and Eq.~\eqref{eq:din} ensure that
the source and the destination is connected to one and only one
POI in the \emph{correct direction}, i.e., via one outgoing edge
and one incoming edge respectively only.  Eq.~\eqref{eq:sin} and
Eq.~\eqref{eq:dout} ensure that there is no incoming edge to the
source and no outgoing edge from the destination.
Eq.~\eqref{eq:out} and Eq.~\eqref{eq:in} constrain that each POI
has at most one incoming and at most one outgoing edge.
Eq.~\eqref{eq:mode} ensures that the chosen mode of travel is at
most one of the $m$ possibilities.  Eq.~\eqref{eq:connectivity}
imposes the \emph{connectivity constraint} by ensuring that a
node, if chosen, has exactly one incoming edge and an outgoing
edge.  Eq.~\eqref{eq:nodechosen1} and Eq.~\eqref{eq:nodechosen2}
imposes the condition that if a node is chosen, corresponding
edges in and out of it must be present as well.
Eq.~\eqref{eq:edgechosen1} and Eq.~\eqref{eq:edgechosen2} ensures
the opposite, i.e., if an edge is chosen, the corresponding nodes
are chosen as well.  Finally, Eq.~\eqref{eq:self} ensures that
there is \emph{no self-loop}.
Eq.~\eqref{eq:subtour} models an important constraint, that of \emph{subtour
elimination}.
Since the optimization is a \emph{maximization} problem, without this subtour
elimination constraint, the solution may include isolated cycles of POIs.
Eq.~\eqref{eq:subtour} prevents it by ensuring that if POI $k$ is visited
immediately after POI $j$, the start time of $v_k$ should be more than that of
$v_j$ plus the time of visit of $v_j$ and the travel time from $v_j$ to $v_k$.

\ignore{

There are five sets of variables in the formulation: $y, m, c, t, z$.
The number of variables of type $y$ that represent POIs is $O(N)$ where $N$ is number of POIs in the city.
\ignore{, and in a multi-day itinerary is $O(N \cdot d)$.}
The number of variables of type $m$ that represent travelling from one POI to another is $O(T.N^2)$ where $T$ is total number of modes of transportation available.
\ignore{, and in a multi-day itinerary is $O(mN^2 \cdot d)$.}
The number of variables of types $c$ and $t$ that represent cost and time
is $O(N) + O(N^2)$.
\ignore{, which scale to $O(N^2 \cdot d)$ in the multi-day case.}
The number of variables of type $z$ that represent connectivity is $O(N^2)$.
\ignore{, and in a multi-day itinerary is $O(N^2 \cdot d)$.}
Hence, the total number of variables in the ILP model is $O(N^2)$ and a careful analysis of all the constraints shows that the total number of constraints is $O(N^2 + |MN| + |N|)$.
\ignore{Hence, the total number of variables in the multi-day ILP model is $O(N^2 \cdot d)$ which boils down to $O(N^2)$ in case of single day itinerary.}

}

There are four sets of variables in the formulation: $y, c, t, z$.  The number
of variables of type $y$ that represent POIs is $O(N)$, where $N$ is number of
POIs in the city.  The number of variables of types $c$ and $t$ that represent
cost and time is $O(T.N) + O(T.N^2)$, where $T$ is the number of modes of
transportation.  The number of variables of type $z$ that represent connectivity
is $O(T.N^2)$.  Hence, the total number of variables in the MILP program is
$O(T.N^2)$ for a single-day itinerary.  For a multi-day itinerary with $D$ days,
the number of variables becomes $O(D.T.N^2)$.  While $N$ is typically less than
$100$, $D$ and $T$ are small constants and, hence, the number of variables is
approximately $10^4$ for most practical scenarios.

