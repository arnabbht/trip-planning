\section{Introduction}

Tourism is one of the most dynamic and rapidly growing sectors of the modern economy. However, planning trips that offer a rich travel experience remains a nontrivial problem facing most tourists. A tourist planning to visit a new city faces the problem of planning a suitable \emph{itinerary} that maximizes her utility under a given cost budget and time budget. Here, an itinerary refers to a sequence of points of interest (POIs), along with the respective time of arrival and departure for each POI such that each POI is visited at most once. With the increasing number of tourists and the increased availability of spatio-temporal data, there is growing research interest in planning the trip itinerary~\cite{li2016travel, gavalas2014survey, sylejmani2011survey}. This problem is important not only for tourists, but also for tour planners that offer personalized trips as business.

Traditionally, most tourist destinations offer a set of pre-defined itineraries that do not necessarily match with the tourists' schedule, cost budget, and preferences. As digital tourism resources and urban mobility platforms grow, tourists expect personalized, optimal, and efficient travel itineraries that meet their needs, constraints, and resources. 
Such itineraries are challenging to construct manually due to the underlying complexity arising due to the number of POIs, varying travel costs and durations, entrance fees, opening hours, and different user interests. This creates an imperative need for intelligent itinerary planning systems that can generate effective and personalized trip itineraries that offer high utility while being cognizant of the requirements of tourists. To this end, although there exist some prior works \cite{chen2014automatic, vanzelst2016itinerary, taylor2018tour, vu2022branch, panagiotakis2024expectation, liu2024personalized, rambha2024optimized, lim2018personalized, bolzoni2014efficient}, they have several limitations. Some key limitations include: (1) None of the existing works allow the tourist to choose from available transportation modes such as only walking, only car, or a hybrid mode allowing usage of multiple transportation modes in the itinerary. (2) Many tourists spend more than a day at a given tourist destination. This scenario ideally calls for a multi-day trip itinerary planning that is aware of opening and closed days of each POI along with opening and closing time for each open day. However, majority of the earlier studies focus only on single-day itinerary planning \cite{taylor2018tour, vu2022branch, panagiotakis2024expectation}.  While it may be possible to combine multiple single day itineraries to generate a multi-day itinerary, it is not guaranteed to be cost-effective (as demonstrated in Sec.~\ref{}). Further, earlier studies paylittle attention to the operational schedule (i.e., operational timings of each open day) of the POIs. Authors of \cite{chen2014automatic, vanzelst2016itinerary, taylor2018tour, vu2022branch, panagiotakis2024expectation, liu2024personalized} does not consider the working days of the POIs and the time windows of the POIs are not addressd by~\cite{chen2014automatic, taylor2018tour, panagiotakis2024expectation} (3) Often there are tourists with specific requirements and preferences based on their priorities and interests. These requirements and preferences must be respected along with the tourist's cost budget and time availability constraints. However, most of the existing itinerary planning solutions ignore this need to \emph{personalize} trip itineraries \cite{chen2014automatic, vanzelst2016itinerary, taylor2018tour, vu2022branch, panagiotakis2024expectation, liu2024personalized, rambha2024optimized, lim2018personalized, bolzoni2014efficient}. (Oswald) To the best of our knowledge, almost all the above work consider either Must-see POIs or Excluded POIs or do not consider both. (4) While travelling based on a planned itinerary, often it is required to update the remaining itinerary based on unexpected delays or early exits from the previously visited POIs. This demands dynamically adjusting the remaining itinerary as deemed necessary. However, to the best of our knowledge, none of the existing works address this requirement.

Motivated by the above limitations, this work addresses the following itinerary planning problem: \emph{Given a tourist who intends to visit a set of POIs spread across a geographical region, how to  identify an optimal itinerary that maximizes the tourist's utility under a specified cost budget and that respects her availability schedule that may span multiple time intervals spread over one or more days?} The itinerary must adhere to the operational schedule (day and timings) of each POI. The itinerary should factor in the tourist's preferred mode of transportation such as only walking, only car, or a hybrid mode that uses both walking and car, as necessary. The cost of the itinerary comprises of two components: (a) transportation cost, i.e., the cost incurred in travelling, (b) visiting cost, i.e., the cost incurred due to entry fees at each POI. The tourist may also specify one or more personalized constraints that include the following: (a) \textbf{Must-see constraints:} These constraints specify a set of POIs that should be necessarily part of the itinerary. (b) \textbf{Ordering constraints:} These indicate relative ordering between two or more POIs in the itinerary. (c) \textbf{Category constraints:} Based on the similarity of the POIs, they may be classified into one or more categories. For example, museums, lakes and churches could be a set of categories.  The category constraints allow a tourist to specify a lower bound and an upper bound on the number of POIs she wants to visit from each category. For instance, a tourist may decide to visit at most one museum and $k$ churches in her itinerary where $1 \le k \le 2$. 

Based on the feedback of previous tourists, each POI is assumed to have a user-rating and recommended visit duration. The utility of a tourist at a given POI depends on the fraction of recommended visit duration she actually spends at the given POI and its average user-rating. The utility of the itinerary is aggregate of the utilities derived at each visited POI. The proposed trip itinerary planning problem allows the tourist to choose from a set of three utility variants that best captures her travel experience. The first variant offers full utility at a POI only if the tourist spends at least the recommended visit duration at the given POI, and zero otherwise. The second variant offers utility that is proportional to the fraction of time that the tourist spends at the given POI w.r.t. its recommended visit duration, provided the tourist  spends at least a minimum specified time. While the first variant can be viewed as a binary step function, the second variant can be viewed as its continuous linear counterpart. The third variant is a multi-step utility function., i.e., a $t$-step utility function where $t \ge 3$. The goal is to maximize the chosen utility variant.

Additionally, if there are unplanned delays or early exits in the earlier part of the itinerary, it should be possible to dynamically update the remaining itinerary based on the remaining cost budget and time budget. Here it is important to note that the reported itinerary not only returns a sequence of POIs to be visited, but also determines the amount of time the tourist spends at each POI (that in turn affects her utility) along with the suitable transportation mode (if more than one transportation modes are available) which in turn affects the travel time and the travel cost of the itinerary.

To address the above problem, this work proposes a novel solution framework,  called \textbf{\trip (TRip Itinerary Planner)}.  Firstly, the problem is modelled as a directed multi-graph $G$ where each node corresponds to a POI, and each edge corresponds to a travel between an ordered pair of POIs using a specific transportation mode such as walking or car. If there are $k$ ($k \ge 1$) transportation modes available between a given pair of POIs, then there are exactly $k$ directed edges between the corresponding pair of nodes in $G$, where each edge corresponds to a specific transportation mode, along with the associated travel cost and travel time. Subsequently, the solution is modelled using a mixed integer linear program (MILP) that returns an optimal itinerary.

The major contributions of this work are summarized as follows:
%
\begin{enumerate}
\item \textbf{Optimal Multi-day Itinerary:} This work proposes a novel multi-day trip itinerary planning solution, named \trip~ that returns an optimal itinerary under the specified cost budget and time budget constraints, while factoring the operational schedule (i.e., operational hours of each open day) of each POI.
\item \textbf{Multimodal Tour:} To the best of our knowledge, this is the first work to consider multiple transportation modes such as only walking, only car, or an hybrid mode that uses both walking as well as car, while planning trip itineraries.
\item \textbf{Personalized Constraints:} The proposed solution framework allows the tourist to specify one or more personalized constraints in the form of must-see constraints, ordering constraints and category constraints, as described earlier.
\item \textbf{Utility Variants:} The \trip~ solution framework allows the tourist to choose an utility variant from a set of three utility variants that best models her travel experience, as discussed above. The itinerary that maximizes the chosen utility variant is reported as the solution.
\item \textbf{Dynamic Rerouting:} This is the first work that allows dynamic adjustment of the remaining itinerary based on unplanned delays or early exits experienced while visiting the previous POIs of the itinerary.
\item Empirical evaluation on several popular tourist destinations confirm the efficacy and efficiency of the proposed solution. 
%
\end{enumerate}
