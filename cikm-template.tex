%% For submission and review of your manuscript please change the
%% command to \documentclass[manuscript, screen, review]{acmart}.
%%
%% When submitting camera ready or to TAPS, please change the command
%% to \documentclass[sigconf]{acmart} or whichever template is required
%% for your publication.
%%
%%
\documentclass[sigconf,authordraft]{acmart}
%%
%% \BibTeX command to typeset BibTeX logo in the docs
\AtBeginDocument{%
  \providecommand\BibTeX{{%
    Bib\TeX}}}

%% Rights management information.  This information is sent to you
%% when you complete the rights form.  These commands have SAMPLE
%% values in them; it is your responsibility as an author to replace
%% the commands and values with those provided to you when you
%% complete the rights form.
\setcopyright{acmlicensed}
\copyrightyear{2018}
\acmYear{2018}
\acmDOI{XXXXXXX.XXXXXXX}
%% These commands are for a PROCEEDINGS abstract or paper.
\acmConference[Conference acronym 'XX]{Make sure to enter the correct
  conference title from your rights confirmation email}{June 03--05,
  2018}{Woodstock, NY}
%%
%%  Uncomment \acmBooktitle if the title of the proceedings is different
%%  from ``Proceedings of ...''!
%%
%%\acmBooktitle{Woodstock '18: ACM Symposium on Neural Gaze Detection,
%%  June 03--05, 2018, Woodstock, NY}
\acmISBN{978-1-4503-XXXX-X/2018/06}


%%
%% Submission ID.
%% Use this when submitting an article to a sponsored event. You'll
%% receive a unique submission ID from the organizers
%% of the event, and this ID should be used as the parameter to this command.
%%\acmSubmissionID{123-A56-BU3}

%%
%% For managing citations, it is recommended to use bibliography
%% files in BibTeX format.
%%
%% You can then either use BibTeX with the ACM-Reference-Format style,
%% or BibLaTeX with the acmnumeric or acmauthoryear sytles, that include
%% support for advanced citation of software artefact from the
%% biblatex-software package, also separately available on CTAN.
%%
%% Look at the sample-*-biblatex.tex files for templates showcasing
%% the biblatex styles.
%%

%\citestyle{acmnumeric}

\usepackage{amsmath}
\usepackage{hyperref}
\usepackage{url}
%\usepackage{tcolorbox} 
%\usepackage{fullpage}
%\usepackage{adjustbox}
%\usepackage{pifont}
%\usepackage{makecell}
%\usepackage{float}  % in preamble
\usepackage{graphicx}
\usepackage{comment}
\usepackage{enumitem}
%\usepackage{floatrow}
\usepackage{textcomp}
\usepackage{wrapfig}
%\usepackage{lscape}
%\usepackage{rotating}
\usepackage{graphicx}
\usepackage{caption}
%\usepackage{upgreek}
\usepackage{gensymb}
%\usepackage{tabularx}
%\usepackage{csquotes}
%\usepackage{pdfpages}
%\usepackage{lipsum}
\usepackage{booktabs}
%\usepackage{array}
%\usepackage{adjustbox}
%\usepackage{amssymb}
%\usepackage{pifont}

\newcommand{\trip}{TRIP}
%\newcommand{\cmark}{\ding{51}} % ✓

%\newcommand{\comment}[1]{}
\newcommand{\ignore}[1]{}

\begin{document}

\title
{Planning the Perfect Itinerary: Optimal Multimodal Tour with  Personalized Constraints and Dynamic Rerouting}

\renewcommand{\shorttitle}{Planning the Perfect Itinerary}

%%
%% The "author" command and its associated commands are used to define
%% the authors and their affiliations.
%% Of note is the shared affiliation of the first two authors, and the
%% "authornote" and "authornotemark" commands
%% used to denote shared contribution to the research.

\author{Anonymous Authors}

\ignore{

\author{Neelu Lalchandani}
\affiliation{%
  \institution{Dept. of Computer Science and Engineering}
  \city{IIT Kanpur}
  \country{India}
  \email{@iitk.ac.in}
}

\author{Priyanshu Jha}
\affiliation{%
  \institution{Dept. of Computer Science and Engineering}
  \city{IIT Kanpur}
  \country{India}
  \email{@iitk.ac.in}
}

\author{Shubhadip Mitra}
\affiliation{%
  \institution{Blue Yonder India Pvt. Ltd.}
  \city{Bengaluru}
  \country{India}
  \email{shubhadip.mitra@blueyonder.com}
}

\author{Arindam Pal}
\affiliation{%
  \institution{TechSoftX Corporation}
  \city{Sydney}
  \state{NSW}
  \country{Australia}
  \email{arindamp@techsoftx.com.au}
}

\author{Oswald Christopher}
\affiliation{%
  \institution{Dept. of Computer Science and Engineering}
  \city{NIT Tiruchirappalli}
  \country{India}
  \email{oswald@nitt.edu}
}

\author{Arnab Bhattacharya}
\affiliation{%
  \institution{Dept. of Computer Science and Engineering}
  \city{IIT Kanpur}
  \country{India}
  \email{arnabb@iitk.ac.in}
}

}

%\renewcommand{\shortauthors}{Trovato et al.}

\begin{abstract}
	%
	Given a tourist who intends to visit a set of points of interest (POIs)
	spread across a geographical region, the \emph{Trip Itinerary Planning
	(TIP)} problem aims to identify an optimal itinerary that maximizes the
	tourist's utility under a specified cost and time budget. An itinerary is
	defined as an ordered sequence of POIs that adheres to the time and budget
	constraints. This problem is not only important for tourists, but also for
	tour planners that offer personalized trips as business. Although there are
	few prior works that have attempted to address the above problem, they allow
	limited flexibility in terms of accommodating multiple transportation modes,
	re-planning the itinerary based on tourist's actual visit duration of the
	POIs and the live traffic situation, customizing the itinerary based on the
	personalized constraints and the utility function chosen by the user. This
	work revisits the TIP problem with the goal of overcoming the above
	limitations and considering more realistic factors.
	%
	In particular, the proposed solution allows the tourist to (1) choose from a
	set of utility function variants that best captures his/her travel
	experience; (2) factor in multiple transportation modes based on available
	cost and time budget; (3) dynamically adjust the remaining itinerary based
	on the actual time spent till the current POI; (4) plan a multi-day
	itinerary that considers the opening and the closing times of the POIs (in
	addition to open and closed days) and user's choice of the start and the end
	time of the trip and the originating POI and the ending POI for each day;
	(5) customize the itinerary by allowing the tourist to choose from a rich
	set of personalized constraints that include must-visit constraints,
	must-avoid constraints, category constraints and ordering constraints. The
	problem is modeled using a directed graph, where the nodes represent the
	POIs and the edges represent the available transportation modes between each
	pair of POIs. Subsequently, this problem is solved using a mixed integer
	linear program (MILP) that returns the optimal itinerary. A comprehensive
	empirical evaluation over a real data set comprising of several popular
	tourist destinations demonstrate the efficacy and efficiency of the proposed
	solution.
	%
\end{abstract}

%%
%% The code below is generated by the tool at http://dl.acm.org/ccs.cfm.
%% Please copy and paste the code instead of the example below.
%%
\begin{CCSXML}
<ccs2012>
 <concept>
  <concept_id>00000000.0000000.0000000</concept_id>
  <concept_desc>Do Not Use This Code, Generate the Correct Terms for Your Paper</concept_desc>
  <concept_significance>500</concept_significance>
 </concept>
 <concept>
  <concept_id>00000000.00000000.00000000</concept_id>
  <concept_desc>Do Not Use This Code, Generate the Correct Terms for Your Paper</concept_desc>
  <concept_significance>300</concept_significance>
 </concept>
 <concept>
  <concept_id>00000000.00000000.00000000</concept_id>
  <concept_desc>Do Not Use This Code, Generate the Correct Terms for Your Paper</concept_desc>
  <concept_significance>100</concept_significance>
 </concept>
 <concept>
  <concept_id>00000000.00000000.00000000</concept_id>
  <concept_desc>Do Not Use This Code, Generate the Correct Terms for Your Paper</concept_desc>
  <concept_significance>100</concept_significance>
 </concept>
</ccs2012>
\end{CCSXML}

\ccsdesc[500]{Do Not Use This Code~Generate the Correct Terms for Your Paper}
\ccsdesc[300]{Do Not Use This Code~Generate the Correct Terms for Your Paper}
\ccsdesc{Do Not Use This Code~Generate the Correct Terms for Your Paper}
\ccsdesc[100]{Do Not Use This Code~Generate the Correct Terms for Your Paper}

%%
%% Keywords. The author(s) should pick words that accurately describe
%% the work being presented. Separate the keywords with commas.
\keywords{Do, Not, Us, This, Code, Put, the, Correct, Terms, for,
  Your, Paper}
%% A "teaser" image appears between the author and affiliation
%% information and the body of the document, and typically spans the
%% page.

\maketitle

\cite{chen2014automatic}

\cite{taylor2018tour}

The bibliography is included in your source document with these two
commands, placed just before the \verb|\end{document}| command:
\begin{verbatim}
  \bibliographystyle{ACM-Reference-Format}
  \bibliography{bibfile}
\end{verbatim}
where ``\verb|bibfile|'' is the name, without the ``\verb|.bib|''
suffix, of the \BibTeX\ file.

Citations and references are numbered by default. A small number of
ACM publications have citations and references formatted in the
``author year'' style; for these exceptions, please include this
command in the {\bfseries preamble} (before the command
``\verb|\begin{document}|'') of your \LaTeX\ source:
\begin{verbatim}
  \citestyle{acmauthoryear}
\end{verbatim}

\ignore{

\section{Acknowledgments}

Identification of funding sources and other support, and thanks to
individuals and groups that assisted in the research and the
preparation of the work should be included in an acknowledgment
section, which is placed just before the reference section in your
document.

This section has a special environment:
\begin{verbatim}
  \begin{acks}
  ...
  \end{acks}
\end{verbatim}
so that the information contained therein can be more easily collected
during the article metadata extraction phase, and to ensure
consistency in the spelling of the section heading.

Authors should not prepare this section as a numbered or unnumbered {\verb|\section|}; please use the ``{\verb|acks|}'' environment.

\section{Appendices}

If your work needs an appendix, add it before the
``\verb|\end{document}|'' command at the conclusion of your source
document.

Start the appendix with the ``\verb|appendix|'' command:
\begin{verbatim}
  \appendix
\end{verbatim}
and note that in the appendix, sections are lettered, not
numbered. This document has two appendices, demonstrating the section
and subsection identification method.

\section{Multi-language papers}

Papers may be written in languages other than English or include
titles, subtitles, keywords and abstracts in different languages (as a
rule, a paper in a language other than English should include an
English title and an English abstract).  Use \verb|language=...| for
every language used in the paper.  The last language indicated is the
main language of the paper.  For example, a French paper with
additional titles and abstracts in English and German may start with
the following command
\begin{verbatim}
\documentclass[sigconf, language=english, language=german,
               language=french]{acmart}
\end{verbatim}

The title, subtitle, keywords and abstract will be typeset in the main
language of the paper.  The commands \verb|\translatedXXX|, \verb|XXX|
begin title, subtitle and keywords, can be used to set these elements
in the other languages.  The environment \verb|translatedabstract| is
used to set the translation of the abstract.  These commands and
environment have a mandatory first argument: the language of the
second argument.  See \verb|sample-sigconf-i13n.tex| file for examples
of their usage.

\section{SIGCHI Extended Abstracts}

The ``\verb|sigchi-a|'' template style (available only in \LaTeX\ and
not in Word) produces a landscape-orientation formatted article, with
a wide left margin. Three environments are available for use with the
``\verb|sigchi-a|'' template style, and produce formatted output in
the margin:
\begin{description}
\item[\texttt{sidebar}:]  Place formatted text in the margin.
\item[\texttt{marginfigure}:] Place a figure in the margin.
\item[\texttt{margintable}:] Place a table in the margin.
\end{description}

%%
%% The acknowledgments section is defined using the "acks" environment
%% (and NOT an unnumbered section). This ensures the proper
%% identification of the section in the article metadata, and the
%% consistent spelling of the heading.
\begin{acks}
To Robert, for the bagels and explaining CMYK and color spaces.
\end{acks}

}

%%
%% The next two lines define the bibliography style to be used, and
%% the bibliography file.
\bibliographystyle{ACM-Reference-Format}
\bibliography{references}

\ignore{

%%
%% If your work has an appendix, this is the place to put it.
\appendix

\section{Research Methods}

\subsection{Part One}

Lorem ipsum dolor sit amet, consectetur adipiscing elit. Morbi
malesuada, quam in pulvinar varius, metus nunc fermentum urna, id
sollicitudin purus odio sit amet enim. Aliquam ullamcorper eu ipsum
vel mollis. Curabitur quis dictum nisl. Phasellus vel semper risus, et
lacinia dolor. Integer ultricies commodo sem nec semper.

\subsection{Part Two}

Etiam commodo feugiat nisl pulvinar pellentesque. Etiam auctor sodales
ligula, non varius nibh pulvinar semper. Suspendisse nec lectus non
ipsum convallis congue hendrerit vitae sapien. Donec at laoreet
eros. Vivamus non purus placerat, scelerisque diam eu, cursus
ante. Etiam aliquam tortor auctor efficitur mattis.

\section{Online Resources}

Nam id fermentum dui. Suspendisse sagittis tortor a nulla mollis, in
pulvinar ex pretium. Sed interdum orci quis metus euismod, et sagittis
enim maximus. Vestibulum gravida massa ut felis suscipit
congue. Quisque mattis elit a risus ultrices commodo venenatis eget
dui. Etiam sagittis eleifend elementum.

Nam interdum magna at lectus dignissim, ac dignissim lorem
rhoncus. Maecenas eu arcu ac neque placerat aliquam. Nunc pulvinar
massa et mattis lacinia.

}

\end{document}
%\endinput

